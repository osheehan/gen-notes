\documentclass[12pt]{article}
\usepackage[letterpaper, portrait, margin=1in]{geometry}
\usepackage[hidelinks]{hyperref}
\usepackage{setspace}
\usepackage{fancyhdr}
\usepackage{graphicx}
\usepackage[T1]{fontenc}
\usepackage{mathptmx}
\usepackage[backend=biber, style=mla]{biblatex}
\pagestyle{fancy}

\addbibresource{ip_state.bib}    

\fancyhf{} % sets both header and footer to nothing}
\renewcommand{\headrulewidth}{0pt}
\pagenumbering{arabic}
\rhead{Sheehan \thepage} 


\def\class{Foundations of Drama I}
\def\prof{Dr.\ Ryan Prendergast}
\def\due{2/18/24}
\def\tw{Thornton Wilder}
\def\w{Wilder}

\fancypagestyle{1stPage}{
    \fancyhf{}
    \renewcommand{\headrulewidth}{0pt} % removes horizontal header line
    \lhead{Owen M. Sheehan \\ \prof{} \\ \class{} \\ \due{}}
    \rhead{ Sheehan 1}
}

\thispagestyle{1stPage}
\begin{document}
\begin{doublespace}
\vspace*{20pt}
    \begin{center}
        \textbf{\large{IP Statement on Author and Production History}}
    \end{center}
% #region Author Biography
    \par \tw\ was born on April 17, 1897, in Madison, Wisconsin to Isabella and Amos P. Wilder.
    \w\ moved between the U.S. and China as a kid due to his father being appointed consul general to Hong Kong, and later Shanghai, by Theodore Roosevelt.
    After graduating from high school, \w\ attended college at Oberlin in Ohio. In 1917, he transferred to Yale University.
    \par While in school \w\ wrote for different school publications, and after graduating, he started writing his first novel, \textit{The Cabala}.
    During this time, \w\ taught French at the Lawrenceville School in Princeton for 4 years, resigning to get his M.A. in French at Princeton.
    \par \w's works all have strong themes of his brand of humanism which Rex Burbank describes as 
    ``Wilder's humanism was akin to the New Humanism in its insistence upon the validity of human values inherited from the cultural past, but his, like [T.S.] Eliot's, had a religious foundation and a sensitivity to the aesthetic as well as the ethical qualities of literature
    \dots [Wilder's] humanism had a religious base to give it an ultimate justification, it, by its very nature, held all dogmas and absolutes in suspension, and was often directed against the excesses of religious belief and puritanical moralism as it was against the coldness of the rationalistic temper'' \autocite*[29]{Thorton}.
% #endregion
% #region Production History
    \par When it comes to the production history of \textit{The Long Christmas Dinner}, we start at Yale University on November 5, 1931, where the play was first produced.
    It was later adapted into an opera by Paul Hindemith, with \w\ writing the libretto. For the sake of this paper, we will mostly just be looking at the play version and not the opera.
    From what I can tell, there were a few TV movies produced based off of the play, however, the only thing that really comes back is an IMDB page with no info other than actors, therefor I'm just going to mention it in passing.
    The next major review for a production of \textit{Christmas Dinner}, is from the LA Times, reviewing a 1988 production by the Pacific Theatre Ensemble in Venice, with author Sylvie Drake opening with a line I think sums up the show really well, ``\tw's ``The Long Christmas Dinner'' is one of the shortest, and sweetest, theatrical meals in the repertory. And one of the more challenging, since this wistful little one-act play (a scant act at that) requires its actors to riffle through 90 years in far less than that many minutes'' \autocite{LA_times}.
    \par When then move to 1993 Broadway production of \textit{Wilder, Wilder, Wilder} which was a compilation of 3 short plays by \tw , \textit{The Happy Journey to Trenton and Camden}, \textit{The Long Christmas Dinner}, and \textit{Pullman Car Hiawatha} respectively.
    After that, there is a bit of a gap until 2014, when the American Symphony Orchestra staged a double production of both the play and opera versions of \textit{The Long Christmas Dinner} at Carnegie Hall. Fun fact, CMU special visiting faculty for Costume Design Olivera Gajic did the costumes for said production.
    Moving on, there is a 2015 off-Broadway production called \textit{A Wilder Christmas} that consisted of \textit{The Long Christmas Dinner} and \textit{Pullman Car Hiawatha}.
    And finally, we come to the most recent major production, which is a run of the show in 2021 and 2022 by the national theatre of Ireland, The Abbey Theatre.


% #endregion




\newpage
\nocite{*}
\printbibliography[
    title ={Works Cited}
]
\end{doublespace}
\end{document}