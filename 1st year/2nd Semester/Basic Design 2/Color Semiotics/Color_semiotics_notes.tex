\documentclass[12pt]{article}
\usepackage[hidelinks]{hyperref}
\usepackage{fancyhdr}
\pagestyle{fancy}
\fancyfoot{}
\rhead{\thepage}
\renewcommand{\headrulewidth}{0pt}
\pagenumbering{arabic}
\usepackage[backend=biber, style=mla]{biblatex}
\addbibresource{colorSemiotics.bib}  

\def\class{Basic Design 2}
\def\prof{Olivera Gajic}

\title{\class{}\\Color Semiotics Notes}
\author{Owen M. Sheehan\\Professor \prof{}}
\date{Spring 2024}

\begin{document}
\maketitle
\tableofcontents
\newpage
    \section{My Modern Met}
    Citation \autocite{ModernMet}
        \begin{itemize}
            \item Most commonly associated with nature today
            \item In Ancient Egypt, green was the symbol of regeneration and rebirth
            \item in the Middle Ages, clothing was used to indicate a person's social rank and profession with green signifying merchants, bankers, and the gentry
            \item due to the rarity and difficulty of green pigment, it came to symbolize wealth
            \item today green symbolizes sustainability and eco friendliness.
            \item Despite many green pigments being toxic, green is still associated with feelings of vitality, freshness, calmness, and revival
        \end{itemize}
    \section{Green in Islam}
        \begin{itemize}
            \item In Islam green has come to symbolize the sky, farming and agriculture, justice, angels, heaven and heavenly beings, the prophet and imams, the moon and water, wisdom and knowledge, faith and firm belief, vitality, youth, and reliance in mysticism \autocite[26]{Islam}
            \item The notion of justice come from a Zoroastrian sect where green was the symbol for struggle against tyranny \autocite[27]{Islam}
            \item In the Qur'an, the color of the heavenly beings clothing has been introduced as green \autocite[27]{Islam}
            \item It is stated in Orad al-Ahbab that ``The Messenger of God preferred green among the colors and the clothes of the heavenly people are green'' \autocite[28]{Islam}
        \end{itemize}
    \section{color symbolism}
        \begin{itemize}
            \item Emblem of freshness, youth, growth, regeneration, and activity \autocite[140]{colorSymbolism}
            \item the Greeks associated green with Aphrodite, the goddess of celestial love, who was born of sea foam
            \item the Greeks also associated green to all the sea deities and held green sacred to the restless sea itself \autocite[140]{colorSymbolism}
            \item In China, green is symbolic of the east, trees, spring, charity, and regeneration \autocite[140]{colorSymbolism}
            \item To Muslims, green is emblematic of Allah \autocite[140]{colorSymbolism}
            \item In India Green represents regeneration \autocite[140]{colorSymbolism}
            \item To Christians, green also symbolizes regeneration \autocite[140]{colorSymbolism}
            \item The notion of regeneration is quite prevalent, due to the ``regeneration'' of plants in the spring \autocite[141]{colorSymbolism}
            \item The meaning of green can also be quite ambivalent, for example, Greeks used it to signify victory as well as defeat and fighting \autocite[141]{colorSymbolism}
            \item it can also mean despair, degradation, and folly \autocite[141]{colorSymbolism}
            \item Green also symbolizes jealously, while jealous can be used as a synonym for green \autocite[141]{colorSymbolism}
            \item to some Scottish clans, green is considered unlucky \autocite[141]{colorSymbolism}
        \end{itemize}



    \newpage
    \printbibliography
\end{document}
    