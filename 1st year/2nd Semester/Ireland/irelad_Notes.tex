\documentclass[12pt]{article}
\usepackage[hidelinks]{hyperref}
\usepackage{fancyhdr}
\pagestyle{fancy}
\fancyfoot{}
\rhead{\thepage}
\renewcommand{\headrulewidth}{0pt}
\pagenumbering{arabic}

\def\class{Modern Ireland}
\def\prof{Aidan Beatty}
\def\th{\textsuperscript{th}}

\title{\class{} Notes}
\author{Owen M. Sheehan\\Professor \prof{}}
\date{Spring 2024}

\begin{document}
\maketitle
\tableofcontents
\newpage
    \section{Week 1}
        \subsection{Wednesday 1/17/24}
            \begin{itemize}
                \item We study history through geography which can be messy due to \\changing borders
                \item Ireland is a simple, single geographical area
                \item The island is split between Northern Ireland (UK) and the Republic of Ireland (EU)
                \item Northern Ireland
                \item from 1801--1922, ``Ireland'' is a legally distinct space within the fluid spaces of the UK and the British Empire
                \begin{itemize}
                    \item The North of Ireland
                    \item The North
                    \item Ulster
                    \item The Six Counties
                \end{itemize}
                \item Ireland
                \begin{itemize}
                    \item The Irish Free State/Saorstát Éireann
                    \item Éire
                    \item The Republic of Ireland
                    \item The Irish Republic
                    \item The South
                    \item The 26 Counties
                \end{itemize}
                \item The further west you go, the less dense the population becomes
                \item Ireland is officially bilingual

                \newpage
                \item Who and Where and the ``Irish Nation''
                \begin{itemize}
                    \item The Irish nation is a quasi-geographical term, but it resists any simple geographical definition
                    \item And we are not studying ``Ireland'', the geographical space, we are studying ``Ireland'', the demographic, social, economic, political, and cultural phenomenon
                    \item ``Irish'' history is inseparable from ``British'' history
                    \item By 1860, the most Irish city in the world, by population, was New York
                    \item today $\frac{1}{5}$ of the population of ``Ireland'' were not born on the island of Ireland
                    \item Ireland, for the entirety of our period, was a part of global \\capitalism, part of broader trends in European and global politics and culture
                    \item ``Ireland'' is fluid and messy
                \end{itemize}
            \end{itemize}
    \newpage
    \section{Week 2}
        \subsection{The Irish Famine}
            \begin{itemize}
                \item For a variety of reasons, european populations grew rapidly from about 1800 onwards
                \item urbanization and industrialization also started to grow from around 1850
                \item potatoes (and meat) are central to these developments
                \item potatoes are a low-labor crop that solve the ``bottleneck'' problem of urbanization
                \item blight starts in belgium and subsequently spreads across greater europe
                \item Why is it so bad in Ireland
                \begin{itemize}
                    \item too many people in rural communities, with entire sections of the Irish population dependent on one single crop
                    \item lack of knowledge about alternative food production
                    \item A spike in crop prices prior to 1846
                    \item The Fungus/Water problem
                    \item patterns of land-ownership and small farms
                    \begin{itemize}
                        \item the Rundale [Roinn Dáil] and Clachan systems of shared land usage
                        \item not all farms/leases are passed on via primogeniture
                    \end{itemize}
                    \item The Corn Laws (1846) and Free Market Liberalism
                    \begin{itemize}
                        \item June 1846, new Liberal government formed in London
                        \item ``The Irish died of political economy'' -John Mitchel
                    \end{itemize}
                    \item ``Malthusian'' conceptions of the Catholic Irish
                    \begin{itemize}
                        \item Did the Irish bring this on themselves?
                        \item Maybe famine, emigration, and mass death will be good in the long run?
                        \item Maltusian Definition - ``Malthusianism is the theory that population growth is potentially exponential, according to the Malthusian growth model, while the growth of the food supply or other resources is linear, which eventually reduces living standards to the point of triggering a population decline''
                    \end{itemize}
                    \item The lack of transport infrastructure
                \end{itemize}
                \item Famine is a ``mechanism for reducing surplus population'' and ``the judgement of God sent the calamity must not be too much mitigated.\ \dots the real evil is with which we have to contend in not the physical evil of the Famine, but the moral evil of the selfish, perverse, and turbulent character of the people'' -Charles Trevelyan (1807-1866), Asst. Secretary to the Treasury and high-ranking civil servant tasked with overseeing response to the famine
                \item British Relief Association gave approx. £400,000 to Ireland (\$ 34m in inflation-adjusted terms)
                \item Myths and Common misconceptions about the Famine
                \begin{itemize}
                    \item Did Ireland export food during the Famine?
                    \begin{itemize}
                        \item Yes, but it also imported a lot of food. The net answer is ``No''
                    \end{itemize}
                    \item Was this an act of genocide?
                    \begin{itemize}
                        \item It probably doesn't meet the level of agency required to be classed as such
                    \end{itemize}
                    \item Did the Irish flee on ``Coffin ships''?
                    \begin{itemize}
                        \item The class divides between who does and does not emigrate, really complicates this
                    \end{itemize}
                \end{itemize}
                \item Long Term Impacts of the Famine
                \begin{itemize}
                    \item An immediate population decrease of about 20%
                    \begin{itemize}
                        \item 1841 census --- 8.18 m
                        \item 1851 census --- 6.55 m
                        \item evenly split between deaths (starvation and starvation-related diseases) and immigration
                        \item 1847 is the worst year for Famine, and high-levels of death continue until 1851
                    \end{itemize}
                    \item Emergence of an Irish diaspora, mainly in settler-colonies of the anglosphere world
                    \item out-migration remains a fact of life in Ireland until 1980s
                    \item The Famine precipitates a linguistic shift from Irish to English
                    \item The famin will become a major ``event'' in Irish popular memory, feeding into a narrative about British cruelty and the dangers of lacking national soverignty, even if this is not fully reflective of sentiments, attitudes and developments at the time
                    \item It precipitates a Devotional Revolution in Irish life
                    \item The Devotional Revolution is linked, in complicated ways, to a broader shift in the class structure and patterns of land-ownership in post-Famine Ireland; a more confident Catholic rural bourgeoisie starts to emerge out of the Famine, they legitimate their status via Catholicism, they see Catholicism and Irishness as linked categories (you can't be one if you're not the other?) and they are the ones who fund the Church. This is the social class that will come to dominate ``Ireland'' by the early 20\textsuperscript{th} century
                \end{itemize}
                \item Devotional Revolution
                \begin{itemize}
                    \item A psychological response to the trauma of the Famine?
                    \item Overseen by Paul Cullen, a cardinal appointed by Rome in 1850
                    \item Cullen sides with ``Ultramontane'' Catholicism rather than ``Gallican'' Catholicism
                    \item A qualitative and quantitative change in the nature of the Catholic priesthood in Ireland
                    \item The ``sins'' of ``drunkenness, women, and avarice'' are purged
                    \item Larkin quotes a report that there are ``few practical Christians'' among the Catholics of Dublin as of 1852
                    \item Only a $\frac{1}{3}$ or so of Catholics attended mass on Sunday prior to 1850
                    \item questionable how many ``Catholics'' were Catholic in strictly doctrinal terms
                    \item The Devotional Revolution was bound up with a transport/infrastructure Revolution
                    \begin{itemize}
                        \item prior to the 1840s, there were not enough roads, railways, or harbors to deliver Famine aid or to minister to congregations
                        \item by c.1875, it was easier to travel around Ireland
                        \item the Church also greatly expanded its capital assets
                    \end{itemize}
                    \item Necessitated the end of ``Stations''
                    \begin{itemize}
                        \item ``Private'' masses are a contradiction in terms, in Catholicism
                        \item A greater level of public surveillance of Catholic populations by the clergy
                        \item more ability to incorporate lower social classes in Catholicism
                    \end{itemize}
                \end{itemize}
                \item The trope/cliché of Famine-era Ireland as a society starkly divided between landlords and tenants
                \item often becomes a moralizing account about cruel landlords and victimized tenants
                \item A very Broad Outline of the Class Structure of Famine-Era Rural Ireland
                \begin{description} 
                    \item[Landlords] Tended to be protestant but not exclusively so
                    \begin{description}
                        \item[Secure Landlords] Will probably survive the Famine
                        \item[Insecure Landlords] Will probably no survive as a ``class''
                    \end{description}
                \end{description}
                \item Tenants
                \begin{description}
                    \item[Secure Tenants] Will probably survive, and even benefit in the long-term, some emigrants come from these families. Religiously mixed
                    \item[Insecure Tenants] Will not probably survive as a ``class'', liable to emigrate or, if they stay, become déclassé. Emmigration often funded by landlords. Mostly ``Catholic'' 
                \end{description}
                \item Sub Tenancy class
                \begin{description}
                    \item[Sub-tenants and landless poor] Not likely to ``survive '', in either meaning of the term. Not likely to emigrate. Almost exclusively ``Catholic''
                \end{description}
                \item Catholicism and the New Irish Catholic Rural Bourgeoisie
                \begin{itemize}
                    \item Late Marriage and Primogeniture
                    \item Catholicism and public morality
                    \item Priests as enforcers of the social order
                    \item The Church as a place to park excess population (as priests and nuns) [alongside emigration]
                \end{itemize}
            \end{itemize}
            \newpage
        \subsection{Irish Diasporas}
            \begin{itemize}
                \item Some quick facts about Irish migration to North America in the 1840s/1850s
                \begin{itemize}
                    \item Two methods of traveling
                    \begin{itemize}
                        \item Conventional passenger ships [can have multiple destinations]
                        \item Empty lumber ships [almost always going to Canada]
                    \end{itemize}
                    \item These are pre-steamboat ships
                    \begin{itemize}
                        \item must follow trade-wind routes
                        \item 5 weeks to Canada, 6 week to U.S.
                        \item Australia takes $3\frac{1}{2}$ months
                    \end{itemize}
                    \item Conditions can vary widely depending on length and ship [which in turn are determined by the wealth of the passenger]
                    \begin{itemize}
                        \item Cramped conditions, poor food supply, disease are all part of this
                    \end{itemize}
                \end{itemize}
                \item Diner's Main Arguments
                \begin{itemize}
                    \item The absence of ``food'' in Irish-American memories and the low impact of Irish fod on American cuisine (in comparison to Italian, Jewish, Chinese, or Mexican migrations)
                    \begin{itemize}
                        \item A function of poverty?
                        \item A reaction to anti-Irish sentiment?
                    \end{itemize}
                    \item The Gendered Nature of post-1847 Irish migration
                    \begin{itemize}
                        \item you can tell a lot about immigrants' goals by looking at \textit{who} immigrates
                    \end{itemize}
                    \item Anti-Irish sentiments and the female caricature of ``Biddy''
                    \item Liguistic racism
                    \item Did county-level identities trump national identity for Irish immigrants?
                \end{itemize}
                \item Some Problems with Diner's Work
                \begin{itemize}
                    \item The problems of ``Bleak Nostalgia''
                    \item The problem of collapsing together vastly different times in Irish history
                    \item What is ``traditional'' Irish food in 1847 (or anytime afterwards)
                    \begin{itemize}
                        \item Irish foodways have been disrupted and remade both by Famine but also by the country's status within the British Empire
                        \item ``Genuine Irish Tea''
                    \end{itemize}
                    \item A broader problem of Irishness and whiteness?
                \end{itemize}
                \item Whiteness
                \begin{itemize}
                    \item Whiteness is a contructed phenomenon; it is also a dynamic and expansive one. Whiteness can expand and contract to include and exclude entire groups of people
                    \item whiteness is ``intersectional''; Irish migrants in the 1840s were seen as not white/off-white, because of their extreme poverty or because of their Catholicism. As they became wealthier and as Catholicism became more accepted in the US, their whiteness stopped being called into question
                    \item but whiteness also cuts across class; as with nationalism, white supremacy encourages all members of its imagined community to assume that they have shared interests due to a shared race or ethnicity (and thus to ignore class differences, gender differences, etc.)
                    \item The State plays a key role in creating and re-creating race
                    \begin{itemize}
                        \item eg, The Naturalization Act of 1790 and the work of the Federal Housing Administration (founded in 1934) formally worked to define who was white or who gained materially by virtue of their whiteness; mass incarceration and contemporary housing practices do the same but in an informal and ostensibly race-blind way
                    \end{itemize}
                \end{itemize}
            \end{itemize}
            \newpage
    \section{Week 3}
        \subsection{Fenianism}
            \begin{itemize}
                \item Zooming out to a much bigger scale of Irish history
                \begin{itemize}
                    \item c.430s --- Christianization of Ireland
                    \begin{itemize}
                        \item Over time, Irish Church becomes autonomous of Rome
                    \end{itemize}
                    \item 1155 --- \textit{Laudabiliter} [praiseworthy] Letter of Pope Adrian IV to Henry Irish
                    \item 1169 --- The ``English'' carry out an ``invasion'' of ``Ireland''
                    \item c.1300 --- Actual English rule has shrunk back to The Pale, the Greater Dublin area
                    \begin{itemize}
                        \item The rest of Ireland is nominally English territory but effectively independent
                    \end{itemize}
                    \item 1518 and 1529 --- Start of the Lutheran and Anglican Reformations
                    \begin{itemize}
                        \item The new danger is that Ireland is a potential ally and/or back-door for French or Spanish Catholic invasions
                    \end{itemize}
                    \item 16\th{} \& 17\th{} Centuries --- Plantations of Ireland
                    \begin{itemize}
                        \item Planting English people into Ireland to serve as the new ruling class
                        \item Earliest plantations are uneven if not a failure
                        \item It is the plantations of 1620s and 1650s that really ``succeed''
                    \end{itemize}
                    \item c.1700 onwards --- Penal Laws
                    \begin{itemize}
                        \item Voting, inheritance, conversion, firearms, religious practive, education and preaching, land-ownership, professions
                        \item Mainly anti-Catholic but also initially targets Presbyterians
                        \item Why don't they work?
                        \item end c.1780s
                    \end{itemize}
                \end{itemize} 
                \newpage
                \item Alternative Timeline for Modern Irish History
                \begin{itemize}
                    \item 1782 --- Founding of Grattan's Irish Parliament: a ``Patriotic'' parliament, for which only Protestants could vote
                    \item 1791 --- Founding of the United Irishmen
                    \item 1798 --- United Irishmen Rebellion
                    \item 1800 onwards --- Low-level agrarian violence remains a constant force in Irish life
                    \begin{itemize}
                        \item Whiteboys, Peep o' Day Boys, Molly Maguires, Captain Rock
                        \item Very ritualized violence, within circumscribed limits carried out by secret societies
                        \item very hard to study
                    \end{itemize}
                    \item 1800/1801 --- Act of Union; Ireland comes under direct British rule
                    \item 1803 --- Robert Emmet leads a failed rebellion
                    \item 1822 --- Establishment of the Royal Irish Constabulary (7 years before the Met. Police in Britain)
                    \item 1823 --- Founding of the Catholic Association
                    \item 1829 --- Election of Daniel O'Connell, Catholic Emancipation Act [O'Connell dies in 1847]
                    \item 1831 --- First national schools established in Ireland (1870 for England and Wales and 1872 in Scotland)
                    \item 1842 --- Founding of the \textit{Young Ireland} rebellion; \textit{Young Ireland} is dissolved next year
                    \item 1858 --- Founding of the Irish Republican Brotherhood [also called the Fenians]
                    \item 1867 --- Failed Fenian Uprising
                    \item 1873 --- Founding of the Home Rule League/Home Rule praiseworthy
                    \item 1878 --- IRB starts the ``New Departure''
                    \item 1882 --- Home Rule Party transformed into Irish Parliamentary Party
                \end{itemize}
                \item \textbf{19\th{} Century Ireland is a place defined by low-level violence, secret societies, militarized policing, \& social engineering}
                \item How do we define these concepts
                \begin{itemize}
                    \item Patriotism
                    \begin{itemize}
                        \item Loyalty
                        \item Pride
                        \item Celebration
                    \end{itemize}
                    \item Nationalism
                    \item Republicanism
                    \begin{itemize}
                        \item Unity
                        \item By the people
                        \item Anti-Monarchist
                        \item Educational
                    \end{itemize}
                \end{itemize}
                \item The Irish Rubublican Brotherhood [The Fenians]
                \begin{itemize}
                    \item Founded in 1857
                    \item Organized as a secret, Oath-bound society
                    \begin{itemize}
                        \item Can be placed into the broader context of secret society violence and earlier Irish nationalism
                        \item But is also a very European phenomenon and is perhaps an importation of American repulbicanism via the Diaspora
                        \item What is the view of the Catholic Chuch on secret, oath-bound societies?
                        \item What are the challenges when studying an organization that remains intentionally secret?
                    \end{itemize}
                    \item The name ``Fenian'' invokes ancient Irish mythology (and a very romanticized ancient Irish warrior masculinity)
                    \begin{itemize}
                        \item Fian(na) is Warrior(s) in Irish
                    \end{itemize}
                    \item IRB Violence
                    \begin{itemize}
                        \item 1866 \& 1870--71 --- Fenian invasion of Canada
                        \item 1867 Rising is poorly organized and a failure
                        \item carries out actions in England also (Manchester Martyrs in Nov. 1867 and Clerkenwell bombing in December)
                        \item 1881--85 --- Fenian Dynamite Campaign
                        \item 1882 --- Irish Invicibles, a breakaway from the IRB, murder two British officials in Phoenix Park in Dublin
                        \item Is this terrorism?
                    \end{itemize}
                \end{itemize}
                \item Irish nationalism in the 19\th{} century develops along two broad strands
                \begin{itemize}
                    \item Physical-force republicanism, that seeks a complete break with British rule and is open to the use of violence
                    \item Constitutional nationalism, more moderate in aims and methods
                    \begin{itemize}
                        \item Both are ``Catholic'', in ideology and membership
                        \item Neither are hermetically sealed from the other
                    \end{itemize}
                    \item Along with unionism (which develops a similar moderation/violence divide), these remain the dominant strands of Irish politics, at least until the end of the 20\th{} century, perhaps up until today
                \end{itemize}
            \end{itemize}
            \newpage
        \subsection{Parnell}
            \begin{itemize}
                \item Charles Stewart Parnell (1846--1891)
                \begin{itemize}
                    \item Born into a wealthy and landed Protestant family
                    \begin{itemize}
                        \item An ``improving landlord''
                    \end{itemize}
                    \item 1875 --- Elected to Parliament for the Irish Parliamentary Party [IPP]
                    \begin{itemize}
                        \item British Parliament is, at this thime, almost completely dominated the Tories and Liberals
                    \end{itemize}
                    \item 1876 --- Parnell makes comments in Parliament that is perceived as pro-Fenian
                    \begin{itemize}
                        \item At the very least, Parnell becomes the leading figure of the radical wing of the IPP
                        \item The more radical interpretation, is that he is an achitect of a Fenian-IPP alliance
                        \item Promotes the policy of ``Obstructionism'', which in time also boosts his profile
                    \end{itemize}
                    \item 1878--1882 --- Land War [will be discussed later]
                    \begin{itemize}
                        \item The high-tide of Parnell's alliances with Fenians
                    \end{itemize}
                    \item 1880 --- Parnell becomes leader of the IPP
                    \item 1882 --- Kilmainham Treaty
                    \begin{itemize}
                        \item Parnell commits himself to only using parliamentary tactics and to an informal alliance with the Liberal Party
                        \item Pulls Fenians into parliamentary politics and weakens the IRB
                        \item It is after 1882 that Parnell is really dominant
                    \end{itemize}
                    \item Accross all of this, Parnell operates via holding the balance of power in the British electoral system
                \end{itemize}
                \item Influenced by American politics (from reading)
                \begin{itemize}
                    \item 296 --- he was definitely following American politics in the press
                    \item 298 --- Sees himself as an heir of Jeffersonian Liberty
                    \item 299 --- Like a lot of contemporary Irish Nationalists, he collapses together the Irish Question with the Slavery Question
                    \item Is this essay just frustratingly vague and inconclusive? 
                \end{itemize}
                \item The Home Rule Bills
                \begin{itemize}
                    \item Government of Ireland Bill, 1886 [First Home Rule Bill]
                    \begin{itemize}
                        \item Would give Ireland its own Parliament
                        \item Westminster would still control external issues (trade, peace treaties, war, etc.) and the RIC
                        \item Defeated by 341 to 311 votes [many Liberals oppose it, for Unionist reasons]
                        \item Even limited Irish autonomy is still too controversial
                    \end{itemize}
                \end{itemize}
                \item Parnell and Catholicism
                \begin{itemize}
                    \item If Ireland is, by the 1870s and 1880s, an intensely Catholic society and one defined by sectarianism, and if Irish nationalism is a Catholic project, how and why does a Protestant become ``The Uncrowned King of Ireland''?
                    \item Are we overstating the severity of the sectarian divide?
                    \item Is he a Catholic Protestant?
                    \item Is this a pragmatic recognition of the religious politics of the broader UK\@?
                    \begin{itemize}
                        \item In general, Parnell works via two unwieldy and ultimately fragile alliances [Moderates/Radicals] and [Catholic/``Not Catholic'']
                    \end{itemize}
                \end{itemize}
                \item The Parnellite/Anti-Parnellite Split
                \begin{itemize}
                    \item Dec. 1890 --- William O'Shea, an Irish-born officer in the British army, files for divorce from his wife, Katherine O'Shea. Parnell is ``named'' in the divorce proceedings
                    \item Parnell and Katherine O'Shea married in June 1891
                    \item According to the rules of Victorian sexual morality, the affair is not the real source of the scandal. The real source is that Parnell was publicly found out
                    \begin{itemize}
                        \item Katherine O'Shea becomes known in the press as ``Kitty'' O'Shea
                        \item It is treated as a shameful secual scandal that the press love to talk about [\textit{occupatio}]
                    \end{itemize}
                    \item 6 October 1891 --- Parnell dies of stomach canccer and coronary heart disease
                \end{itemize}
                \item The Split:
                \begin{itemize}
                    \item The party splits into 45 anti-Parnellite MPs vs. 27 Parnellite MPs in December 1890
                    \item A Catholic/``secular'' split? a ``conservative''/``radical'' split?
                    \item The IPP becomes far less of a mass movement, more imperialist, more conservative in subsequent years
                    \item The split is the defining issue for the next generation
                    \item Does it cause a retreat from ``politics'', which in turn brings about a new kind of politics?
                \end{itemize}
                \item Parnell as the Irish Moses
                \begin{itemize}
                    \item Parnell leads Ireland to the ``promised land'' of Independence; a ``Joshua'' that will actually enter the promised land is still forthcoming
                    \item Parnell ``brought Ireland within sight of the Promised Land. The triumph of the national cause awaits another time, and another Man'' --- R. Barry O'Brien.\ \textit{The Life of Charles Stewart Parnell (1898)}
                    \item What does it mean to compare a political leader to Moses? Is this an unqualified positive statement?
                    \item The leader might be great, but the nation is corrupt, weak-willed, longs to return to slavery [i.e. is afraid of Freedom], and they worship false gods, not the one true God
                    \item The nation needs a great leader, because it is riddled with flaws
                \end{itemize}
                \item \textit{Portrait of the Artist as a Young Man} (1916)
                \begin{itemize}
                    \item A portrait of middle-class Catholic Dublin, the novel depicts anti-Parnellite politics as narrow-minded Catholic bigotry
                    \begin{itemize}
                        \item Parnellite politics are depicted as more progressive
                        \item But also notice that some people just want to be rid of this divide
                    \end{itemize}
                \end{itemize}
                \item The Home Rule Bills
                \begin{itemize}
                    \item Government of Ireland Bill, 1886 [First Home Rule Bill]
                    \begin{itemize}
                        \item Would give Ireland its own parliament
                        \item Westminster would still control external issues (trade, peace treaties, war, etc.) and the RIC
                        \item Defeated by 341--311 votes [many Liberals oppose it, for Unionist reasons]
                        \item Even limited Irish autonomy is still too controversial
                    \end{itemize}
                    \item Government of Ireland Bill, 1893 [Second Home Rule Bill]
                    \begin{itemize}
                        \item Largely the same proposals as 1886 Bill, though with the added element that Irish MPs would also continue to sit in Westminster
                        \item Passes House of Commons by 347--304
                        \item Defeated in the House of Lords [the unelected, Tory-dominated upper house] by 419--41 votes
                    \end{itemize}
                    \item The defeat of the Second Bill, along with the repercussions of the Parnellite Split, ends the possibilities of Home Rule for a generation
                    \begin{itemize}
                        \item 3\textsuperscript{rd} Home Rule Bill passes in 1912, delayed by House of Lords for two years, never implemented
                        \item 4\th{} Home Rule Bill passes in 1920, but only really covers what is today Northern Ireland; rest of Ireland is covered by the Anglo-Irish Treaty of 1922
                    \end{itemize}
                \end{itemize}
            \end{itemize}
            \newpage
    \section{Week 4}
        \subsection{The Land War (1879--1882)}
            \begin{itemize}
                \item Michael Davitt (1846--1906)
                \begin{itemize}
                    \item 
                    \item Born in Mayo, on the west coast of Ireland, at the outset of the Famine
                    \item Family is evicted from their farm in 1850
                    \begin{itemize}
                        \item He claimed that he could remember this formative moment as an adult
                    \end{itemize}
                    \item Family relocated to Lancashire, where he was raised
                    \item Began work in a mill at age 9, lost his right arm at age 11
                    \begin{itemize}
                        \item He received no compensation but was gifted an education by a local philanthropist 
                    \end{itemize}
                    \item Began to move in Chartist circles and joined the IRB in 1865 (aged 19)
                    \begin{itemize}
                        \item by 1867/68, he was working full-time for the IRB
                        \item Arrested in May 1870 on charges of treason, incitement to insurrection, arm trafficking
                        \item Sentenced to 15 years hard labor
                        \item Released in December 1877
                    \end{itemize}
                    \item By 1878, he is back in Ireland to help with the ``New Departure'' [the alliance between the IRB and the IPP] and to contribute to the Land War
                \end{itemize}
                \item After the Land War
                \begin{itemize}
                    \item Davitt becomes a much more cosmopolitan figure than most Irish nationalists
                    \item Visits Egypt and Palestine in 1885
                    \item Tours Australia and New Zealand in 1895
                    \item Visited South Africa during the Boer War (1899--1902)
                    \begin{itemize}
                        \item By the time his book predicting a Boer victory is published, the British have already defeated the Boers
                    \end{itemize}
                    \item Visited Kishinev in the aftermath of the Kishinev Pogrom of Easter, 1903
                    \begin{itemize}
                        \item His book on Kishinev, which compares the plight of Russian Jews to that of the Irish, is published by the Jewish Publication Society of America
                    \end{itemize}
                    \item Speaks out against antisemitic attacks on the small Jewish community in Ireland in 1904
                    \begin{itemize}
                        \item Describing his funeral in 1906: ``There were many touching, and some significant message of regret. There was a wreath from the Jewish community of Dublin, in grateful remembrance of Davitt's efforts on behalf of the one race which has suffered more than the Irish'' [Francis Sheehy-Skeffington.\ \textit{Michael Davitt: Revolutionary Agitator and Labour Leader (1909)}]
                    \end{itemize}
                \end{itemize}
                \item What happened in the Land War and its aftermath [According to Davitt]?
                \begin{itemize}
                    \item A leaderless uprising of the peasantry
                    \item A complete revolution in land ownership in Ireland
                    \item The beginnings of ``a rational principle of state socialism'' in Ireland
                    \item The Land Act of 1891 provides £33m [£3.5b/\$4.4b] to buy out landlords in Ireland and allow tenants to buy their own holdings
                    \item another £112m [£11.26b/\$14.29b] provided under a 1903 Land Act
                    \item Strengthens the case for Home Rule, at least according to Davitt
                    \item A very nationalistic account: ``men of the Irish race, scattered by eviction and evils of unsympathetic rule'' have carried out a successful struggle for ``the Celtic Fatherland \dots the repossession of the soil of the country''
                \end{itemize}
                \item Land Ownership in post-Famine Ireland
                \begin{itemize}
                    \item Eviction orders are common [perhaps 50\% of all farmers received one in two decades after 1850] but actual implementation of an eviction is rarer
                    \item Arrears are common
                    \item In the very western parts of the country, small, rented land-holdings [less than 20 acres] are still common
                    \item Some landlords live in precarious circumstances, particularly those reliant on low-yield, low-rent tenants
                    \item A new class of ``graziers'' (export-oriented cattle farmers with large holding → 50 acres --- that they often still rent) is emerging
                    \item There is a general switch away from crops to animals in Irish agriculture
                    \item Very large landed estates do continue to exist
                    \begin{itemize}
                        \item 20,000 acre estates are not uncommon
                        \item Richard Berridge in Galway owned 117,000 Acres [7.6\% of the county]
                    \end{itemize}
                \end{itemize}
                \item Basic Chronology of the Land War
                \begin{itemize}
                    \item Spring 1879 --- brief return of Potato Blight in the context of a global economic depression
                    \item April 1879 --- Davitt leads a public meeting in Mayo demanding land reform
                    \begin{itemize}
                        \item Davitt's hope is to build an alliance of tenant farmers, shopkeepers and clergy
                        \item Clergy initially refuse to join
                        \item Protest meetings continue to be held into the summer, mainly in Mayo and Galway
                    \end{itemize}
                    \item August 1879 --- Land League of Mayo is founded
                    \item October 1879 --- The Irish National Land League is founded
                    \begin{itemize}
                        \item Leadership positions are taken by IPP figures, Fenians (who don't disclose this membership), clergy
                        \item The league's slogan is ``The Land for the People''
                        \begin{itemize}
                            \item Slogan might mean total land nationalization [which was Davitt's goal]
                            \item Small tenant farmers want land redistribution
                            \item Large tenant farmers want to just become owner-occupiers
                            \item Parnell is a landlord; he opposes a more radical interpretation of the slogan
                        \end{itemize}
                        \item Clerical leadership plays a moderating/conservative role
                    \end{itemize}
                    \item September 1880 --- The Land League ``ordered'' Charles Cunningham Boycott, the agent for an absentee landlord in Mayo, to reduce his rents by 25\%; when he refused, the League orchestrated a collective local campaign to shun him and refuse to do any kind of business with him (and to refuse to above the new 75\% rent) 
                    \item The League's goals are defined as the Three F's
                    \begin{itemize}
                        \item Fair Rent, Fixity of Tenure, Free Sale
                    \end{itemize}
                    \item August 1881 --- British government passes the Land Act
                    \item October 1881 --- Leaders of the Land League are arrested and placed in Kilmainham Gaol
                    \begin{itemize}
                        \item They issue a ``No Rent Manifesto'', demanding a cessation of rents until their release
                        \item The unity needed to pursue this is undercut by the Land Act
                    \end{itemize}
                    \item The 1881 Land Act essentially instituted the Three F's via a system of judicial rent review and dual proprietorship
                    \begin{itemize}
                        \item Government also agrees to fund mortgages that can cover three quarters of the purchase price of a farm
                        \item Only wealthy tenant farmers availed of this and only small numbers
                        \item The 1891 and 1903 Land Acts, offering far more money, have a much bigger impact
                    \end{itemize}
                \end{itemize}
                \item Constructing Irish National Identity
                \begin{itemize}
                    \item Control of the soil of Ireland becomes a central nationalist concern and ``Farmers'' gain a privileged status in nationalist ideology
                    \item The violence of the Land War and the goals of the Land League are sites of contestation
                    \begin{itemize}
                        \item Agrarian agitation takes place within circumscribed rules about what is or is not permissible
                        \item It is a contest over land that ultimately does not challenge private property rights
                    \end{itemize}
                    \item The League portrays itself as a vehicle for exploited tenants, but its most active membership are slightly further up the socio-economic ladder and more aspirational in their goals than this imagery suggests
                \end{itemize}
                \item The Land War in Galway is concentrated in the eastern parts of the county, where farm sizes are slightly larger and more directly assimilated into larger commodity networks and food-supply chains
                \item The poorest parts of Galway are all to the west of Galway City
                \item The Ladies Land League --- Contesting Gender during the Land War
                \begin{itemize}
                    \item Founded in Jan. 1881, dissolved in August 1882
                    \item Fanny Parnell, sister of Charles Stewart Parnell, was a leading figure
                    \begin{itemize}
                        \item Being close to a male leader is often the advantageous position from which nationalist women can act; though her brother was opposed to this
                    \end{itemize}
                \end{itemize}
            \end{itemize}
\end{document}