\documentclass[12pt]{article}
\usepackage[letterpaper, portrait, margin=1in]{geometry}
\usepackage[hidelinks]{hyperref}
\usepackage{setspace}
\usepackage{fancyhdr}
\usepackage[T1]{fontenc}
\usepackage{mathptmx}
\pagestyle{fancy}
\usepackage[backend=biber, style=mla]{biblatex}
\addbibresource{cga.bib}    

\fancyhf{} % sets both header and footer to nothing}
\renewcommand{\headrulewidth}{0pt}
\pagenumbering{arabic}
\rhead{Sheehan \thepage} 
\fancypagestyle{1stPage}{
    \fancyhf{}
    \renewcommand{\headrulewidth}{0pt} % removes horizontal header line
    \lhead{Owen M. Sheehan \\ Tina Cafasso \\ Interpretation and Argument \\ 10/08/23}
    \rhead{ Sheehan 1}
}

\thispagestyle{1stPage}

\begin{document}
    \begin{doublespace}

        \begin{center}
            \vspace*{6pt}
            Comparative Genre Analysis
        \end{center}

    \vspace{-18pt}
    \par%Intro Paragraph
        The articles ``I tweet honestly, I tweet passionately: Twitter users, context collapse, and the imagined audience'' as well as ``How Real Are You on Facebook''
        both deal with a users authenticity online. Where ``I tweet honestly'' is written for an academic audience, ``How Real Are You on Facebook'' is more intended for a wide audience.
        While they both share the same ideas about authenticity, the use of tone, format, and sourcing are quite different, albeit with the occasional slight similarity. 
    \par%1st supporting
        The first major difference between the two artic;es is how they approach citations. In ``How Real Are You on Facebook'', it's charitable to say that author Sophie Goodman uses citations.
        However unlike most cases of online articles not using sources, at least anecdotely, this seems more of a choice of readability, rather than a way to control access to information.
        One way we can be sure of this is that she does mention the authors that she is paraphrasing. Another point of interest is that she does link to academic journals in three places.
        The first two instances come from linking to ``I tweet honestly'' where Goodman introduces language that is explained in said article,
        ``The ethnographer danah boyd \dots, one of the earliest researchers of social lives online,
        refers to ``\underline{\href{http://con.sagepub.com/content/14/1/13.short?rss=1&ssource=mfr}{social convergence}}'' in social networking sites. Social convergence, she argues, occurs when multiple social worlds merge.
        This results in ``\underline{\href{http://nms.sagepub.com/content/13/1/114}{context collapse}},'' meaning social media sites bring together different social contexts simultaneously'' \autocite{facebook}. 
        The underlined words link to ``I tweet honestly'', and are therefor, in a sense, are a type of citation. However, unlike actual citations, it fails to give more information on where in the text this definition comes from, 
        making it a lot harder to parse the data.
        The third instance of a academic article being linked comes from a link to ``The Rules of Facebook friendship''\footfullcite{face_friends}. 
        Marwick and boyd, on the other hand, use citations extensively as one would expect from a academic journal, take for example the following quote,
        ``Self-conscious identity performances have been analyzed in internet spaces like social network sites (boyd, 2007; Livingstone, 2005), blogs (Hodkinson and Lincoln, 2008; Reed, 2005), 
        dating sites (Ellison et al., 2006) and personal homepages (Papacharissi, 2002; Schau and Gilly, 2003)'' \autocite[115]{tweet}.
        I think this snippet perfectly outlines the differences in how these two articles cite sources. In Marwick and boyd's article, we see that every time they slightly mention a separate source there is a citation, whereas Goodman just alludes to it.
    \par%2nd supporting
        The next area where they differ is in how they format the information respectively. In ``How Real Are You On Facebook'', Goodman formats the information in a way that is more intended for a wide audience.
        It does not have an abstract or headings, or anything like that. The article conforms more to the norm for online articles as opposed to academic articles.
        Where ``I tweet honestly'' is organized with an abstract and headings, in part to help a reader find a specific piece of information by being able to quickly find a section they're looking for, ``Facebook'' does not.
        If an academic article wasn't organized as such, it would be much harder to read, as it turns into a massive mountain of text with no guide. Another thing that ``I tweet honestly'' does is format direct quotes from participants in a clearly different style.
        In my opinion, however, I feel like an online article as short as ``How Real Are You on Facebbok'' would not benefit from headings since there is such little information comparatively. And on top of that, Goodman does give a line of space between sections.s
    \par%3rd supporting
        The two articles are actually quite close in tone. However, there are slight differences. Take for example, Sophie Goodman takes a far more casual approach to presenting fairly similar information.
        When talking about how Facebook is introducing features to sort friends and how people are reacting to this, she says, `` I suspect it's simply too time-consuming for people to maintain'' \autocite{facebook}.
        With this example, it's far clearer how she is willing to inject her own conclusions into the argument, rather than sythesizing from other data or her own research, that is to say, her article is far more anectdotal.
        Marwick and boyd, on the other hand, give examples from participants and other research when coming to conclusions, after writing
        ``Self-censorship can be a useful technique in the face of an imagined audience that includes parents, employers, and significant others. Some respondents assumed anyone could potentially read their tweets, making it impossible to discuss controversial or personal topics'' \autocite*[125]{tweet}.
        They follow up with quotes from people participating to support their argument. This gives them, at least to me, a tone that is more authoritative as opposed to Goodman, since their arguments are backed up with examples.
    \par%conclusion
        All in all, while the articles do present pretty similar info, with Goodman even pulling info from ``I tweet honestly'', they are presented in two very different ways. That isn't to say that there isn't a place for articles such as ``How Real Are You on Facebook''.
        It's important to have articles that introduce this new information to people in a more digestable format. It wouldn't be a good idea to give someone the entirety of ``I tweet honestly'' and expect them to read a 15 page article with no background knowledge.
        However, it would be nice if Goodman gave a list of places to find more information for those who are inclined to know more. Other than that I think it's a wonderful article to get started with.
    \newpage \printbibliography
    \end{doublespace}
\end{document}
