% !TeX root = proposal_draft.tex
\documentclass[12pt]{article}
\usepackage[hidelinks]{hyperref}
\usepackage{fancyhdr}
\usepackage[backend=biber, style=mla]{biblatex}
\addbibresource{proposal_draft.bib}
\pagestyle{fancy}
\fancyfoot{}
\rhead{Sheehan \thepage}
\fancypagestyle{1stPage}{
    \fancyhf{}
    \renewcommand{\headrulewidth}{0pt} % removes horizontal header line
    \lhead{Owen M. Sheehan \\ Tina Cafasso \\ Interpretation and Argument}
    \rhead{ Sheehan 1}
}
    
\renewcommand{\headrulewidth}{0pt}
\pagenumbering{arabic}

\begin{document}
\thispagestyle{1stPage}
\hfill
\tableofcontents
\newpage
    \section{proposal draft 11/07/23}
        \subsection{Introduction}
            In the last 2 decades, social media has become such big part of our social lives, to the point that for many people, it has replaced traditional social interaction.
            It has become an even greater tool for people who have a large social following, with many people making their living off of sites such as YouTube and Instagram.
            When it comes to personalities online, their credibility and authenticity is highly important and can effect their livelihoods.
            Ergo, it is of utmost importance to retain this image.
        \subsection{Synthesis}
            To this point my research question is \textbf{``How does the average person perceive authenticity online, do they truly believe what they see, or is it viewed as a facsimile of authenticity''}
            There are differing viewpoints as to what constitutes authenticity. The first viewpoint is that authenticity is about revealing, or seeming to reveal personal information.
            The other viewpoint is that authenticity is about having a consistent brand image overtime, leading to consitent opinions over time.
        \subsection{Methodology}
            To figure out how different groups of people interact with authenticity online I want to send out a survey to different groups of people both on and off campus.
            This survey will feature both multichoice and open ended questions as well as a space to put any additional comments.
        \subsection{Hypothesis} 
            I think that the anwers will fall into two different categories when it comes to different factors. When it comes to accounts for people who are not famous, or semi-famous, it is more that they are more authentic, since they do not have a large outward facing profile.
            However, on the other hand, when it comes to those with a large, or semi-large public profile, it is more likely, at least to me, that a lot of the posts that seem to be authentic, are more a façade, or in someway disengenious. Now, it most be noted that I am quite cynical and not a good control for things like this, but still.
    \newpage
    \nocite{*}
    \printbibliography[
        heading= bibintoc,
        title ={Annotated Bibliography}
    ]
\end{document}