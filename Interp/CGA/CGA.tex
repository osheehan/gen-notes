\documentclass[12pt]{article}
\usepackage[letterpaper, portrait, margin=1in]{geometry}
\usepackage[hidelinks]{hyperref}
\usepackage{setspace}
\usepackage{fancyhdr}
\usepackage[T1]{fontenc}
\usepackage{mathptmx}
\usepackage[backend=biber, style=mla]{biblatex}
\pagestyle{fancy}
\addbibresource{cga.bib}    

\fancyhf{} % sets both header and footer to nothing}
\renewcommand{\headrulewidth}{0pt}
\pagenumbering{arabic}
\rhead{Sheehan \thepage} 
\fancypagestyle{1stPage}{
    \fancyhf{}
    \renewcommand{\headrulewidth}{0pt} % removes horizontal header line
    \lhead{Owen Sheehan \\ Tina Cafaso \\ Interpretation and Argument \\ 9/29/23}
    \rhead{ Sheehan 1}
}

\thispagestyle{1stPage}

\begin{document}
    \begin{doublespace}

        \begin{center}
            \vspace*{6pt}
            Comparative Genre Analysis
        \end{center}

    \vspace{-18pt}
\par%Intro Paragraph
        The articles ``I tweet honestly, I tweet passionately: Twitter users, context collapse, and the imagined audience'' as well as ``How Real Are You on Facebook''
        both deal with a users authenticity online. Where ``I tweet honestly'' is written for an academic audience, ``How Real Are You on Facebook'' is more intended for a wide audience.
        While they both share the same ideas about authenticity, the use of tone, format, and sourcing are quite different, albeit with the occasional slight similarity. 
\par%1st supporting

    \end{doublespace}
\end{document}
