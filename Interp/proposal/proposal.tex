\documentclass[12pt]{article}
\usepackage[letterpaper, portrait, margin=1in]{geometry}
\usepackage[hidelinks]{hyperref}
\usepackage{setspace}
\usepackage{fancyhdr}
\usepackage[T1]{fontenc}
\usepackage{mathptmx}
\usepackage[backend=biber, style=mla]{biblatex}
\newcommand{\mycomment}[1]{}
\pagestyle{fancy}
\addbibresource{proposal.bib}    
\fancyhf{} % sets both header and footer to nothing}
\renewcommand{\headrulewidth}{0pt}
\pagenumbering{arabic}
\rhead{Sheehan \thepage} 
\fancypagestyle{1stPage}{
    \fancyhf{}
    \renewcommand{\headrulewidth}{0pt} % removes horizontal header line
    \lhead{Owen M. Sheehan \\ Tina Cafasso \\ Interpretation and Argument \\ 12/04/23}
    \rhead{ Sheehan 1}
}


\thispagestyle{1stPage}
\begin{document}
\begin{doublespace}
\vspace*{20pt}


% #region Intro ===========================================================
    This essay will look at the current research regarding authenticity online and then propose research on how the average person perceives authenticity online. 
    To begin, we will look at the current research consensus and look on how it can be expanded.  
    Previous research has been done on how people interact with authenticity online, however, most of these articles get their data from systematic reviews of certain peoples accounts online, 
    while others instead ask participants via social media about their thoughts on social media. 
    I believe that this can be a flawed way to go about getting data for such studies as it is reliant on people who participate in social media, which may omit some more cynical views of social media. 
    As such, I'm going to try and answer the question ``How does the average person perceive authenticity online, do they truly believe what they see, or is it viewed as a facsimile of authenticity''. 
% #endregion Intro =========================================================
% #region synthesis ===========================================================
\par There are a few concepts that are quite important when talking about authenticity online. 
    Firstly, there is the idea of ``Context Collapse'', which is when multiple different audiences get flattened into one \autocite[122]{MB1}.
    Secondly, there is the idea of the ``lowest common denominator'', which is a term created by Bernie Hogan which is the idea that one posts content online that is socially suitable to the people who may not be the intended audience of a post, but can still see the post. \Autocite*[383]{Hogan1}
    These two ideas form the basis for how we look at people's posting behaviours online. It must also be noted that there are a few distinct types of accounts that you will see online, those being personal, professional, and corporate. This essay looks at how people interact with the first two for the most part, as when it comes to authenticity online, corporate accounts don't really matter as most of what they post is marketing materials or ads.
    \par Bernie Hogan uses Erving Goffman's dramaturgical approach, which he describes as ``a metaphorical technique used to explain how an individual presents an ``idealized'' rather than authentic version of herself. The metaphor considers life as a stage for activity. Individuals thus engage in performances, which Goffman defines as ``activity of an individual which occurs during a period marked by his continuous presence before a particular set of observers and which has some influence on the observers''\autocite[22]{Goffman}. This continued presence allows individuals to tweak their behavior and selectively \dots \ give off details, a process he termed `impression management.'''\autocite[378]{Hogan1}
    to show that a person's ``performance'' online is catered to an audience that is causing said person's performance to ``consist of the selective details that one presents in order to foster the desired impression alongside the unintentional details that are given off as part of the performance'' \autocite[378]{Hogan1}.
    However, as pointed out by Pitcan et al, social media complicates this as the different ways we present ourselves to separate audiences, or ``code switching'', isn't easily done since the separate audiences are collapsed together. On top of that, the impression management literature based off Goffman, fails to adequately account for the structural differences that affect individuals options and risks. \autocite*[164]{Pitcan1}

% #endregion synthesis =========================================================










% #region Notes Bernie Hogan (Hogan1)
\mycomment{
    \begin{itemize}
        \item Everything we post online leaves a trace
        \item we also use impression management in real life
        \item Looking at Goffman's Dramaturgical approach, we can see that people curate their images in front of others to present an idealized version of themselves rather than an authenic version
        \item People either act in the "front stage" where they present an idealized version of themselves or in the "back stage" where the work to keep up appearances is done, the true self
        \item Within the Dramaturgical approach, there is an audience that a performant is catered to, where an actor put up a front, the front consists of selective details that once presents in order to foster the desired impression
        \item the actors front is involves continual adjustment of self-presentation based on the presence of others
        \item individuals put ip a specific front and modify it based on the observation of others
        \item 
        \item 
    \end{itemize}
}
% #endregion

\newpage \printbibliography
\end{doublespace}
\end{document}
