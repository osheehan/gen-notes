\documentclass[12pt]{article}
\usepackage[letterpaper, portrait, margin=1in]{geometry}
\usepackage[hidelinks]{hyperref}
\usepackage{setspace}
\usepackage{fancyhdr}
\usepackage[T1]{fontenc}
\usepackage{mathptmx}
\usepackage[backend=biber, style=mla]{biblatex}
\newcommand{\mycomment}[1]{}
\pagestyle{fancy}
\addbibresource{proposal.bib}    
\fancyhf{} % sets both header and footer to nothing}
\renewcommand{\headrulewidth}{0pt}
\pagenumbering{arabic}
\rhead{Sheehan \thepage} 
\fancypagestyle{1stPage}{
    \fancyhf{}
    \renewcommand{\headrulewidth}{0pt} % removes horizontal header line
    \lhead{Owen M. Sheehan \\ Tina Cafasso \\ Interpretation and Argument \\ 12/06/23}
    \rhead{ Sheehan 1}
}


\thispagestyle{1stPage}
\begin{document}
\begin{doublespace}
\vspace*{20pt}

% #region Intro ===========================================================
\par This essay will look at the current research regarding authenticity online and then propose research on how the average person perceives authenticity online. 
    To begin, we will look at the current research consensus and look on how it can be expanded.  
    Previous research has been done on how people interact with authenticity online, however, most of these articles get their data from systematic reviews of certain peoples accounts online, 
    while others instead ask participants via social media about their thoughts on social media. 
    I believe that this can be a flawed way to go about getting data for such studies as it is reliant on people who participate in social media, which may omit some more cynical views of social media. 
    As such, I'm going to try and answer the question ``How does the average person perceive authenticity online, do they truly believe what they see, or is it viewed as a facsimile of authenticity''. 
% #endregion Intro =========================================================
% #region synthesis ===========================================================
\par There are a few concepts that are quite important when talking about authenticity online. 
    Firstly, there is the idea of ``Context Collapse'', which is when multiple different audiences get flattened into one \autocite[122]{MB1}.
    Secondly, there is the idea of the ``lowest common denominator'', which is a term created by Bernie Hogan which is the idea that one posts content online that is socially suitable to the people who may not be the intended audience of a post, but can still see the post. \Autocite*[383]{Hogan1}
    These two ideas form the basis for how we look at people's posting behaviours online. It must also be noted that there are a few distinct types of accounts that you will see online, those being personal, professional, and corporate. This essay looks at how people interact with the first two for the most part, as when it comes to authenticity online, corporate accounts don't really matter as most of what they post is marketing materials or ads.
\par Bernie Hogan uses Erving Goffman's dramaturgical approach, which he describes as ``a metaphorical technique used to explain how an individual presents an ``idealized'' rather than authentic version of herself. The metaphor considers life as a stage for activity. Individuals thus engage in performances, which Goffman defines as ``activity of an individual which occurs during a period marked by his continuous presence before a particular set of observers and which has some influence on the observers''\autocite[22]{Goffman}. 
    This continued presence allows individuals to tweak their behavior and selectively \dots \ give off details, a process he termed `impression management.'''\autocite[378]{Hogan1}
    to show that a person's ``performance'' online is catered to an audience that is causing said person's performance to ``consist of the selective details that one presents in order to foster the desired impression alongside the unintentional details that are given off as part of the performance'' \autocite[378]{Hogan1}.
    However, as pointed out by Pitcan et al, social media complicates this as the different ways we present ourselves to separate audiences, or ``code switching'', isn't easily done since the separate audiences are collapsed together. On top of that, the impression management literature based off Goffman, fails to adequately account for the structural differences that affect individuals options and risks. \autocite*[164]{Pitcan1}
\par I think it's appropriate to look at how we analyze micro-celebrites online and the self-branding they use and how that mindset can be seen in the ``everyday person''.
    In Chapter 3 of Alice Marwick's \textit{Status update: Celebrity, publicity, and branding in the social media age}, micro-celebrity is defined as being, ``a state of being famous to a niche group of people, but it is also a behavior: the presentation of oneself as a celebrity regardless of who is pating attention'' \autocite*[114]{Marwick1}.
    One way to achieve this presentation is through self-branding, which in the following chapter Marwick describes as being ``primarily a series of marketing strategies applied to the individual. It is a set of practices and a mindset, a way of thinking about the self as a salable commodity that can tempt a potential employer'' \autocite*[166]{Marwick2}.
    In \textit{`Meat, Mask, Burden': Probing the contours of the branded `self'}, Alison Hearn points out that creating a brand image requires ``creating a detachable, saleable image or narrative, which effectively circulates cultural meaning'' \autocite*[198]{Hearn1}.
    As opposed to self-branding in the real world, the internet ``idealizes transparency ... [and] expect(s) [microcelebrities] to be more available and more ``real'' than stars of the screen or stage'' \autocite[114]{Marwick1}.
\par To satisfy this expectation, microcelebrities ``creat[e] a persona, produc[e] content, and strategically appeal to online fans by being ``authentic'' '' \autocite[114]{Marwick1}.
    This persona of authenticity, can actually be quite contentious as authenticity is a subjective word, especially when it comes to online audiences, ``Of course, Authenticity is a social construct ... we are interested not in an absolute sense of authenticity, but in what Twitter ... considers `authentic' '' \autocite[119]{MB1}.
    Authenticity on the internet usually takes 2 forms, either direct interaction with ``fans'', or through the sharing of personal information \autocite[114]{Marwick1}.
    This creates a schism however, as ``revealing personal information is seen as a marker of authenticity, but is strategically managed and limited'' \autocite[127]{MB1}.
    This becomes a tricky balancing act as that management ``can be seen as \textit{inauthentic}. When asked to describe `authenticity' on Twitter, respondents lpaced it in direct opposition to strategic self-promotion'' \autocite[127]{MB1}.
    Reading all of this, I wanted to see if this skepticism applied not only to micro-celebrities, but also the average person, hence the question ``How does the average person perceive authenticity online, do they truly believe what
    they see, or is it viewed as a facsimile of authenticity''.
% #endregion synthesis =========================================================
% #region method ===========================================================
\par To answer this question, I will be using a survey to gauge people's answers to how they interact with authenticity online, as well as finding out what they consider ``authentic''.
    Questions will include: what social media they use, what kind of accounts they follow, If they think professional or personal accounts are authentic and why, as well as what they consider authentic.
    \underline{\href{https://forms.gle/sjiDcxBzhiiDFehy5}{Link to survery}}.
% #endregion method =========================================================
% #region Hypothesis ===========================================================
\par In my opinion, which I know is quite cynical, I think that a majority of people will say that all accounts are inauthentic, citing that most people have some kind of image they want to project about themselves.
    Now, that is a quite cynical view, however, I also believe that the data will show the percent of people saying personal accounts are inauthentic will be smaller than the group saying professional accounts are inauthentic.
\par the information gathered won't contribute to much since I don't think a statistically large enough group of people will answer the survey. 
    However, if that does happen, it will just be another data point showing that people, in general, are not authentic online, in one form or another
% #endregion hypothesis ===========================================================



\newpage \printbibliography
\end{doublespace}
\end{document}
