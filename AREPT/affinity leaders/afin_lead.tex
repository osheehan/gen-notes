\documentclass[12pt]{article}
\usepackage[hidelinks]{hyperref}
\usepackage{fancyhdr}
\usepackage[backend=biber, style=mla]{biblatex}
\addbibresource{afin_lead.bib}
\pagestyle{fancy}
\fancyfoot{}
\rhead{\thepage}
\renewcommand{\headrulewidth}{0pt}
\pagenumbering{arabic}


\begin{document}
    \section{notes}
        \begin{itemize}
            \item \citefield{blackpast}{journaltitle}
            \begin{itemize}
                \item Born as Alice Herndon, 1916, in Charleston, SC
                \item Parents separated in 1925
                \item moved to grandmothers home of Harlem, NYC who encoured her to write and explore the arts
                \item quit highschool 2 years in, worked low paying jobs while becoming involved in theatre scene
                \item In the 1930's, she both married and divorced Alvin Childress, who was the father to her daughter, Jean R. Childress
                \item She later married musician Nathan Woodard in 1957
                \item Joined Harlem's American Negro Theatre (ANT) where she \\worked as an actress, stage director, personnel director, and \\costume designer for 11 years
                \item While at ANT she fought for for off-broadway union contracts that would assure advanced pay for actors.
                \item As an acress, she appeared in many New York productions, including \textit{Natural Man} (1941), \textit{Rain} (1948), and \textit{The Emperor's Clothes} (1943)
                \item \textit{Anna Lucasta} (1944) transferred to Broadway garnering Childress a Tony nomination
                \item Wrote her first play \textit{Florence} in 1949, which reflects many themes characterized in Childress's later writings such as black female empowerment, \\ interracial politics, working-class life and attacks on black stereotypes
				\item with the off-Broadway performances of \textit{Just a Little Simple} (1950) and \textit{Gold through the Trees} (1952), Childress became the first professionally produced black female playwright
				\item at the end of the '55-'56 off-Broadway season, Childress's \textit{Trouble in Mind} won an Obie Award for Best Original Play, making Childress the first black woman to be awarded the honor, by the end of her career she had written over a dozen plays
				\item One of her most famous works, \textit{A Hero Ain't Nothin' but a Sandwhich} (1973), discusses difficult social issues such as racism, drug use, teen pregnancy, and homosexuality, this novel helped launch her career as a young adult novelist
				\item the novel was adapted into a screenplay in 1978 with Childress writing two other young adult novels, an adult novel, and a collection of short stories
				\item Several of her works caused controversy, with some networks refusing to televise the 1969 production of \textit{Wine in the Wilderness} and the 1973 production of \textit{Wedding Band: A Love/Hate Story in Black and White}. On top of that many school districts and libraries banned \textit{A Hero Ain't Nothin' but a Sandwhich}
				\item Childress received many awards and grants such as: a Rockefeller grant, a graduate medal from the Radcliffe Institute for Independent Study, the Radcliffe Alumnae Graduate Society Medal for Distinguished Achievement, and a Lifetime Career Achievement Award from the Assosciation for Theatre in Higher Education
            \end{itemize}
			\item \Citeauthor{round}
			\begin{itemize}
				\item as a child, Childress spent hours at the public library
				\item Her Grandmother, Eliza White, enjoyed telling stories and encouraged this talent in her granddaughter, together they would make up stories about people they watched from their window
				\item she did not attend complete high school or attend college, she was entirely self-educated, thanks to her grandmother's influence and her own passion for learning.
				\item As a teenager, she saw a Shakespeare play that sparked her interest in theatre.
				\item her first mentor was Venezuela Jones, who ran the Federal Theatre Project's Negro Youth Theatre and was the first Black woman playright Childress met.
				\item W.E.B. Du Bois, Founder of The NAACP and a leading black scholar also inspired Childress, particularly her interest in Africa.
				\item His wife, Shirley Graham Du Bois encouraged Childress to write
				\item ANT paid little to no money so Childress worked a variety of low-paying jobs outside of Theatre
				\item Her first play \textit{Florence} was written overnight for ANT, which she later left to focus on playwrighting
				\item Outside of writing plays, Childress also engaged in real-world political activism.
				\item She worked with the Committee for the Negro in the Arts (CNA), as well as fighting for theatre artists' rights to recieve advances and guaranteed pay for union actors in Off-Broadway productions.
				\item She also wrote a column in the progressive Black newspaper \textit{Freedom} using the persona ``Mildred,'' a domestic worker who shared her experiences of racism
				\item she also taught classes at the Jefferson School of Social Science, a Marxist institute for adult Education
				\item She and Shirley Graham Du Bois founded Sojourners for Truth and Justice, which was a radical Black women's civil rights group that fought against lynching, the rape of Black Women by white men, Jim Crow, South African apartheid, and sexism.
				\item Her association with these left-wing organizations put her on the FBI's surveillance list for many years, however, she was eventually cleared of any association with the Communist Party
				\item Her book \textit{A Hero Ain't Nothin' but a Sandwhich} (1973) which explored the struggle of Black youth in the inner city, recieved many awards, such as, The American Library Association's Best Young Adult Book, the Lewis Carrol Shelf Award and a Jane Addams Award
				\item She also recieved many academic awards later in life. Radcliffe College, where she had been an Assciate Scholar from 1966 to 1968, awarded her an Alumnae Graduate Society Medal for Distinguished Achievement in 1984 
				\item She recieved an Honorary Degree from the State University of New York at Oneonta and an Honorary Doctorate of Fine Arts from State University of New York, both in 1990. On top of this, she recived a Lifetime Achievement Award from the Association for Theatre of Higher Education in 1993.
			\end{itemize}
        \end{itemize}
	\newpage
	\section{script}
		\begin{itemize}
			\item Alice Childress, was born in 1916 as Alice Herndon in Charleston, South Carolina
			\item When she was 9 her parents separated and she went to live with her grandmother, Eliza White, in Harlem, New York
			\item Eliza was interested in storytelling and passed this interest to her granddaughter, and together they would make up stories about the people they watched from their window.
			\item As a teenager Alice saw a Shakespeare play that instilled a passion for theatre in her
			\item Childress did not finish High School or attend college, she was entirely self educated thanks to her grandmother's influence and her own passion for learning.
			\item Her first mentor was Venezuela Jones, who ran the Federal Theatre Project's Negro Youth Theatre and was the First Black female playwright that Childress had met
			\item Another early influence for her was W.E.B. Du Bois, the founder of the NAACP and a leading black scholar, and his wife, Shirley Graham Du Bois, with the couple particularly inspiring her interest in Africa and ecouraging her to write, respectively
			\item She eventually joined Harlem's American Negro Theatre, or ANT, where she worked as an actress, stage manager, directer, and designer, among other things, for 11 years. 
			\item During this time she was seen in many New york productions, such as Natural Man in 1941, the Emperor's Clothes in 1943 and Rain in 1948. She also appeared in the play Anna Lucasta in 1944 which garnered her a Tony Nomination when it transfererd to Broadway.
			\item Even though she was doing shows at ANT, the pay was meager or, at times, non-existant, so Childress took many low-paying jobs outside of theatre to support her family.
			\item Moving on to her playwrighting, In 1949 she wrote her first play Florence, which tackled issues relating to black female empowerment, interracial politics, working class life, and attacks on black stereotypes, overnight for ANT, however she would later leave ANT to focus on playwrighting.
			\item The next few years saw the off-broadway performances of her plays Just a little simple and gold through the tress in 1950 and 1952 respectively. This made Childress the first professionally produced black female playwright.
			\item On top of this, in the 1955 to 56 Off-broadway season, Childress's Trouble in mind won an Obie Award for Best Original Play, making her the first black woman to earn such an award
			\item Outside of playwrighting, Childress was also quite politicaly active.
			\item she worked with the Committee for the Negro in the Arts, fought for theatre artists' rights to recieve advances and guaranteed pay for union actors in off-broadway productions
			\item she also wrote a column for the progressive black newspaper, Freedom, using the persona of Mildred, a domestic worker who shared her experiences with racism. on top of this, she also taught classes at the Jefferson School of Social Sciences, a Marxist institute for adult education
			\item She concurrently founded Sojourners for Truth and Justice, a organization that fought against lynching, the rape of Black Women by white men, Jim Crow, South African Apartheid, and sexism, with Shirley Graham Du Bois.
			\item Due to her association with these left-wing association, Childress waas put on the FBI's surveillance list but was eventually cleared of any connection with the Communist particularly
			\item In addition to her playwrighting, Childress was also an acclaimed author, with her 1973 book \textit{A Hero Ain't Nothin' But a Sandwhich}, which explored the struggle of Black youth in the inner city, recieving many awards such as The American Library Assosciation's Best Young Adult Book, The Lewis Carol Shelf Award, and a Jane Addams Award.
			\item While it did recieve many awards, it also stirred some controversy, with many school districts and libraries banning it.
			\item \textit{A Hero Ain't Nothin' but a Sandwhich} kicked off Childress's novel career, with her going on to write two other young adult novels, an adult novel, and a collection of short stories
			\item Alice recieved many academic awards in her time as well, recieving an Alumnae Graduate Society Medal for Distinguished Achievement in 1984 from Radcliffe College where she was an Associate Scholar from 1966 to 1968.
			\item she also recieved an Honorary Degree from the State University of New York at Oneonta and a Honorary Doctorate of Fine Arts from the State University of New York, both in 1990, on top of this, she recieved a Lifetime Achievement Award from the Assosciation for Theatre in Higher Education in 1993
		\end{itemize}
    \newpage
    \printbibliography
\end{document}