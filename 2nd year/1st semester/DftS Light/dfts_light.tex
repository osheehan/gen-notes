\documentclass[12pt]{article}
\usepackage[hidelinks]{hyperref}
\usepackage{fancyhdr}
\pagestyle{fancy}
\fancyfoot{}
\rhead{\thepage}
\renewcommand{\headrulewidth}{0pt}
\pagenumbering{arabic}

\def\class{Design for the Stage: Lighting}
\def\prof{Mary-Ellen Steebins}

\title{\class{} Notes}
\author{Owen M. Sheehan\\Professor \prof{}}
\date{Spring 2024}

\begin{document}
\maketitle
\tableofcontents
\newpage

    \section{Week 1}
        \bigskip
            \begin{itemize}
                \item lighting has to make sense
                \item you experience light through time
            \end{itemize}
        \smallskip
            \begin{itemize}
                \item Four questions for the semester
                \begin{enumerate}
                    \item Selective visibility: who/what are we seeing?
                    \item Revelation of Form: how are we seeing?
                    \item Composition: what else is there?
                    \item Tone: What does it feel like?
                \end{enumerate}
            \end{itemize}
        \smallskip
            \begin{itemize}
                \item 5 Physical properties of light
                \begin{description}
                    \item[Intensity] Intensity is relative, one light can be brighter than the other without affecting the other
                    \item[Angle] Relationship between the light and what is lit, i.e.\ relative coordinates. You are more predisposed to notice angle
                    \item[Color] Made up of the relationship between the color of the light and the color of the object
                    \item[Shape/Form/Quality] How the actual beam of light is seen
                    \item[Movement/time] The transitions we see within the light. The movements are the cueing
                \end{description}
            \end{itemize}



\end{document}