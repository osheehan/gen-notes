\documentclass[12pt]{article}
\usepackage[hidelinks]{hyperref}
\usepackage{fancyhdr}
\pagestyle{fancy}
\fancyfoot{}
\rhead{\thepage}
\renewcommand{\headrulewidth}{0pt}
\pagenumbering{arabic}

\def\class{Foundations of Drama II}
\def\prof{Dr.\ Amanda Olmstead}
\def\th{\textsuperscript{th}}

\title{\class{} Notes}
\author{Owen M. Sheehan\\Professor \prof{}}
\date{Spring 2024}

\begin{document}
\maketitle
\tableofcontents
\newpage
    \section{Week 1}
        \subsection{Tuesday 8/27/24}
        \bigskip
            \begin{description}
                \item[Historiography] The study of historical writing; the narrativization of history
            \end{description}
        \bigskip
        \subsection{Thursday 8/29/24}
            \bigskip
            \begin{itemize}
                \item What is ``performance''
                \begin{itemize}
                    \item On stage
                    \item Liveness
                    \item Audience
                    \item Story (communicated)
                    \item Intention to be shared
                    \item Ephemeral
                    \item Transmission of memory
                    \item Spiritual
                    \item Religious
                    \item Ritual
                    \item Personal
                    \item Transmission
                    \item Memory
                    \item Frame
                \end{itemize}
                \item Richard Schechner (77)
                \begin{itemize}
                    \item ``a ``performance'' may be defined as all the activity of a given participant on a given occasion which serves to influence in any way any of the other participants''
                    \item Is vs. As
                \end{itemize}
                \pagebreak
                \item Seven Spheres of performance
                \begin{itemize}
                    \item to entertain
                    \item to create beauty
                    \item to make or change identity
                    \item to make or foster community
                    \item to heal 
                    \item to teach or persuade 
                    \item to deal with the sacred ad the demonic
                \end{itemize}
                \item Performance Studies
                \begin{itemize}
                    \item ``Performance is everywhere --- we study how it works'' -NYU PS Dept.
                    \item Schechner --- ``Performance means: never for the first time. It means: for the second to the nth time. Performance is `twice-behaved behavior'\,''
                    \item Performances function as vital acts of transfer, transmitting social knowledge, memory, and a sense of identity through reiterated, or what Schechner has called `twice behaved behavior'
                    \item `Performance,' on one level, constitutes the object/process of analysis in performance studies, that is, the many practices and events --- dance, theatre, ritual, political rallies, funerals --- that involve theatrical, rehearsed, or conventional/event-appropriate behaviors
                    \item To say something \textit{\underline{is}} a performance amounts to an ontological affirmation \dots
                    \item On another level, performance also constitutes the methodologgical lens that enables scholars to analyze events as performance 
                    \item To understand these as a performance suggests that performance also functions as an epistemology
                \end{itemize}
            \end{itemize}
            \bigskip
    \section{Week 2}
        \bigskip
        \subsection{Tuesday}
            \bigskip
            \begin{itemize}
                \item Embodiment
                \begin{itemize}
                    \item Representation
                    \item Memories? a body?
                    \item Staying power? object?
                    \item Existing through the body
                    \item Communication thru body 
                    \item Visual|Physical
                    \item Tangible|Visceral
                    \item Personify?
                    \item \textbf{Emodiment}
                    \begin{description}
                        \item{\textbf{1.}} A tangible or visible form of an idea quality, or feeling
                        \item{\textbf{2.}} The representation or expression of something in a tangible or visible form.
                    \end{description}
                \end{itemize}
                \item Ephemerality
                \begin{itemize}
                    \item Memories
                    \item Staying power
                    \item Protection
                    \item Recording
                    \item Documentation
                \end{itemize}
                \item Field/Language
                \begin{itemize}
                    \item Gaps
                    \item Bias 
                    \item Recognition
                    \item Translation
                \end{itemize}
                \item Performance studies can contribute to our understanding of \\ Latin America --- and hemispheric --- performance traditions by \\ rethinking 19\th\ century disciplinary and national boundaries and by focusing on embodied behaviors
                \item \dots can expand the theoretical scope \dots that has, due to its context,\\ focused more on the future and ends of performance than on its historical practice. 
                \item The Archive
                \begin{itemize}
                    \item ``\,`Archival' memory exists as documents, maps, literary texts, letters, archaeological remains, bones, videos, films, CDs, all those items supposedly resistant to change'' (19)
                    \item What changes over time is the value, relevance, or meaning of the archive, how the items it contains get interpreted, even embodied
                \end{itemize}
            \end{itemize}
\end{document}