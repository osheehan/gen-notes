\documentclass[12pt]{article}
\usepackage[hidelinks]{hyperref}
\usepackage{fancyhdr}
\pagestyle{fancy}
\fancyfoot{}
\rhead{\thepage}
\renewcommand{\headrulewidth}{0pt}
\pagenumbering{arabic}

\def\class{Foundations of Drama II}
\def\prof{Dr.\ Amanda Olmstead}

\title{\class{} Notes}
\author{Owen M. Sheehan\\Professor \prof{}}
\date{Spring 2024}

\begin{document}
\maketitle
\tableofcontents
\newpage
    \section{Week 1}
        \subsection{Tuesday 8/27/24}
        \bigskip
            \begin{description}
                \item[Historiography] The study of historical writing; the narrativization of history
            \end{description}
        \bigskip
        \subsection{Thursday 8/29/24}
            \bigskip
            \begin{itemize}
                \item What is ``performance''
                \begin{itemize}
                    \item On stage
                    \item Liveness
                    \item Audience
                    \item Story (communicated)
                    \item Intention to be shared
                    \item Ephemeral
                    \item Transmission of memory
                    \item Spiritual
                    \item Religious
                    \item Ritual
                    \item Personal
                    \item Transmission
                    \item Memory
                    \item Frame
                \end{itemize}
                \item Richard Schechner (77)
                \begin{itemize}
                    \item ``a ``performance'' may be defined as all the activity of a given participant on a given occasion which serves to influence in any way any of the other participants''
                    \item Is vs. As
                \end{itemize}
                \pagebreak
                \item Seven Spheres of performance
                \begin{itemize}
                    \item to entertain
                    \item to create beauty
                    \item to make or change identity
                    \item to make or foster community
                    \item to heal 
                    \item to teach or persuade 
                    \item to deal with the sacred ad the demonic
                \end{itemize}
                \item Performance Studies
                \begin{itemize}
                    \item ``Performance is everywhere --- we study how it works'' -NYU PS Dept.
                    \item Schechner --- ``Performance means: never for the first time. It means: for the second to the nth time. Performance is `twice-behaved behavior'\,''
                    \item Performances function as vital acts of transfer, transmitting social knowledge, memory, and a sense of identity through reiterated, or what Schechner has called `twice behaved behavior'
TODO finish line below 
                    % \item `Performance,' on one level, constitutes the object/process of analysis in performance studies, that is, the many practices and events --- dance, theatre, ritual, political rallies, funerals --- that involve theatrical, rehearsed,
                    \item To say something \textit{\underline{is}} a performance amounts to an ontological affirmation \dots
                    \item On another level, performance also constitutes the methodologgical lens that enables scholars to analyze events as performance
TODO there is another line to put after this look in the slides
                \end{itemize}
            \end{itemize}
\end{document}