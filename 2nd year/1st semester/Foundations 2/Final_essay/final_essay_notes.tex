\documentclass[12pt]{article}
\usepackage[hidelinks]{hyperref}
\usepackage{fancyhdr}
\usepackage[backend=biber, style=mla]{biblatex}
\pagestyle{fancy}
\fancyfoot{}
\rhead{\thepage}
\renewcommand{\headrulewidth}{0pt}
\pagenumbering{arabic}
\addbibresource{final_essay.bib}    

\def\class{Class}
\def\prof{Prof}

\title{\class{} Notes}
\author{Owen M. Sheehan\\Professor \prof{}}
\date{Spring 2024}

\begin{document}
\maketitle
\tableofcontents
\newpage

    \section{source 1 \autocite{bruceAndrewBruceMusical}}
        \begin{itemize}
            \item Digico unveiled the d5 live console at the 2002 Plasa show in london
            \item Andrew Bruce approached Digico about using the console for theatre
            \item DiGiCo MD James Gordon and his team start off by going to Chitty Chitty Bang Bang which at the time was using a Cadac J-Type, and the operator ran 200 sound cues
            \item Bruce said of the operator ``She was up and down the console following every line --- and they said, we had no idea you did all this, we always thought theatre was quite easy''
            \item article: ``What had shocked the DiGiCo team was the line-by-line mixing technique used by theatre sound engineers. This entails learning the full show script and managing mic levels and EQ for every player --- as well as any sound effects --- to ensure that only active mic channels are open and the remainder fully muted''.
            \item ``With potentially 35 people on stage, every one wearing an omnidirectional mic and all facing in different directions, with a PA that is only five metres away, albeit facing in a different direction, you cannot afford to leave any unused mics up at all. So you're not just following every line, you are killing stone dead every mic that is not literally in use at this point. Mics do not dip, they go completely mute, otherwise you have potentially 50 mic sources all contributing enormous amounts of audio contamination''.
            \item ``The most rudimentary one is just dipping the least important mic, as long as the actors are close enough together. You will get some sort of balance that way, as long as you dip the mic on the person with the stronger voice. But that might only be for a few words, and then as they move away or turn away, you've got to be ready to bring the other mic back''.
            \item ``This is beyond the scope of any automation because they may not turn away on the same word, so the operator has to be watching what they're doing and the balancing is absolutely manual''.
            \item ``A digital console's ability to map channels to the surface offers two immediate and dramatic benefits. It simultaneously pulls groups of channels together under an operator's hands instead of leaving them spread across a massive console and allows the other channels to be `hidden', reducing the number of theatre seats that have to be sacrificed to accommodate it. In addition, storing channel settings allows what DiGiCo has termed Aliases (originally coined to reflect its relevance to the theatre) to be created to accommodate changes in EQ made necessary by the use of hats''.
            \item ``I told James early on in the D5T development, that the one thing all designers were requesting and the single feature thing that would ingratiate DiGiCo to all theatre sound mixers is to recognise and provide for the fact that, if you include understudies, there are always four people playing the same principal part, and to store settings for every single one of those actors. And future players --- if the play runs for years and years, those actors will have been replaced by another four, only to have one of the original four return some years later because he's made it big in the world of TV and has come back to do a guest spot. And that's not just for actors, but for musicians as well.''
            \item ``Being able to store all the alias settings for an unlimited number of people and access them literally during the show without having to stop or call a halt in any way, makes DiGiCo a de facto standard for theatre. It took a while because it's a really complicated bit of programming, but they did just that''.
            \item ``The fact that we could phone through and describe what was happening --- they got it, they altered it, they sent it back to us and we loaded it, all in the course of one day --- encapsulates DiGiCo to me''.
            \item ``With the SD platform, all of the things that Bruce, the other sound designers and DiGiCo had developed were an integral part of the desk's software, including access to a session at a much deeper level than on the D5. Among other things, this allows on-line access to a show's session programming without interfering with a performance, using an additional surface or computer''.
        \end{itemize}
    \newpage
    \section{source 2 \autocite{bruceTheatreSoundDigital}}
        \begin{itemize}
            \item This article is Andrew Bruce recounting the story of the stage and the mixing console when it comes to his company: Autograph Sound.
            \item ``In 1973, there were no loudspeakers, mixers, radio mics or miniature mics that were made specifically for theatre use. And there were only three legal radio frequencies that we could use. We had to learn how to persuade manufacturers to see things our way, which was almost impossible to do because we had no money to commission anything --- manufacturers' first question is always, `how many will you order if we put time into this'\,''
            \item When Bruce worked on \textit{A Chorus Line} in 1976, the sound designer, Abe Jacob, also known as ``The Godfather of Theatre Sound'' had chosen to mix on a Trident Fleximix.
            \item The Fleximix was an early and rudimentary modular mixer aimed at the live market. The main feature was a flexible footprint with different modules that could be bolted together for different numbers of channels and busses.
            \item They used the Fleximix for a few shows until Andrew-LLoyd Webbers \textit{Cats} when they switched to a Midas TR, which was originaly designed for the National Theatre and the console had an output matrix in it.
            \item In the meantime the console company Cadac, which also made consoles focused at live sound, had gone out of business and was bought up by Clive Green from Morgan Studios.
            \item When Autograph Sound was doing \textit{Little Shop of Horrors} at the Comedy (now Harold Pinter Theatre) they were given the specific brief that they couldn't use a space more than 4 seats, therefor they commissioned the new Cadac to create a console that fit in the space. It took 4 weeks for Cadac to design, build, and deliver the console. This showed the team that there was a manufacturer that saw theatre as a recognisable market
            \item ``It was 1983 and shows were becoming complex enough that the actual operation required some kind of assistance, and computers were reasonably commonplace. Clive Green commissioned a chap called Derek Dearden, who was a professor of computing at London University, to build a rudimentary computer that could remember control group assignments. You could program a series of cues that brought the elements that you wanted to the central VCA section, so that the operator wasn't scooting up and down a huge console trying to find things in the dark. That was the beginning of a real sea change in the way mixing in the theatre developed --- the idea that things should come to you, and that required computer help''.
            \item ``Meanwhile, we had started a programme with Clive to design a series of standard consoles for the hire company [Autograph] that could be used for any musical. This idea of building consoles specifically for each show couldnt last forever. Luckily, Starlight lasted 21 years so we got our money back. Incidentally, the computer --- just a card with wires hanging out of the back of it --- only failed once during that time''.
            \item ``So we had started to develop with Clive this range of consoles that embodied all the various things we had learned up to this stage --- modularity of frames, output matrix, balanced busing, all those things that were, by now, deemed totally indispensable. The simplest of these was the A-Type console, which was designed without any automation as that was still a bit experimental and added an unwelcome level of expense. We put the first one on Les Mis in 1985.
            \item That particular console ran eight shows a week for 15 years before being replaced by an automated J-Type two years before the show moved from the Palace to the Queen's Theatre in 2002. While the A-type was totally manual, the J-Type that replaced it for the last two years had VCA assistance from the computer, which instantly simplified a very busy three-and-a-half hour mix.
            \item In the intervening years, we had commissioned the next series, the E-Type, from Clive, which had expanded facilities --- VCAs and programmable control groups became a standard feature so we could start to automate properly. We started developing our own software to control the assignment of control groups, but Clive also had his own versions most of which --- I'm bound to say, were much inferior. Developing our own software allowed us to expand on various extras, like having active legend strips sitting above the control group faders that displayed the names of individuals or groups of people who were assigned to them on a cue-by-cue basis.
            \item It was an ever-expanding brief and widening set of features that we were working on over the years. Miss Saigon in 1989 was the first show that had the legend strips on an E-Type, so it was all coming together.
            \item Nevertheless, these were still big old analogue consoles that were taking up more and more space. Cameron Macintosh in particular wasted no opportunity in pointing out how much money he was losing because of the number of seats we took up, and could we not work towards the same sort of thing that automation boys and the lighting boys had done? It was an interesting point --- but we were struggling with the old analogue-digital debate --- which sounds best? --- which is not something that is relevant to automation or lighting, and which only the most astute producer would understand. But we were definitely at the point where we were having to argue the toss with Cameron and other producers saying, “analogue is still absolutely the only way to go and analogue requires space”. There was no assignability; we can't bury channels and bring them to the surface, they all have to be there simultaneously. It does mean that the operator has to do a lot of moving around during the show fixing things, but right now there is no viable alternative.
            \item We started having to dig our heels in with the producers, who were the ones giving us the jobs, so we had to look for counter arguments, my favourite of which was: “well yes, it's all very well saying you're losing money because of the seats we've taken out but that only applies if you're 100 per cent full all the time, which is not necessarily the case”. Sadly for my argument, it was exactly the case with most of Cameron's shows at the time.
            \item As a result of this relentless pressure from producers, we were constantly on the lookout when the first digital consoles started appearing. We listened to them and tried to make sense of their user interface from our rather specialised point-of-view then, without exception, rejected them for our own use because of their lack of features and their sound. We continued down the route with analogue and we had a lucky break when I was approached in 1992 out of the blue by Chas Brooke and Nigel Olliff, who were both then with BSS, to see if I was interested in joining in a development programme with Soundcraft''.
            \item So we did, we installed the first Soundcraft Broadway at the Prince Edward theatre in the stalls on the right-hand side of the aisle with a Cadac J-Type on the left-hand side of the aisle at the Prince Edward. Which unfortunately meant that we had to have two completely separate cable infrastructures because all the racks for the Broadway were under the stage, while there was a complete alternative set of wiring to the Cadac in the front of house.
            \item But the Broadway never actually controlled any of the audio on the show. It fell over so frequently that, almost on the first day, we changed to the Cadac and never changed back. We let them stay there while they systematically tried to bug fix it, but the networking technology of the time was far too slow. It could successfully handle a single microphone, even as many as five, but once even medium amounts of data started flying around the network between the stage racks and the front-of-house surface, it fell over instantly.
            \item That took us to 2002, which was the year when DiGiCo first showed their D5 Live at the Plasa show in London. I remember being told by a colleague, Bobby Aitken, that I should go along and take a look, because this was the next step in digital consoles. At this point we'd pretty much written off anything that existed in the digital console world as being not much use for musical theatre. We understood the direction that conventional software had taken but, to be quite honest, I didn’t understand how the system of libraries and scope could possibly be relevant to us because the way we work is very, very specific and is quite unknown outside of theatre \dots
            \item I hung around at the end of one of his demonstrations, took a look and said [to DiGiCo MD James Gordon] that I'd done a bit of work with Soundcraft and Cadac, and had a mine of information on an application they may not have considered. Would they be interested in evolving something specifically for the theatre?
        \end{itemize}
        \newpage
    \section{source 3 \autocite{coulesMixingEvolutionKey2020}}
        \begin{itemize}
            \item Microphones and loudspeakers came from wireless telegraphy
            \item the main purpose of those early system was to amplify and propogate the spoken word
            \item ``The advent of rock `n' roll, the development of the electric guitar, and the need to perform in larger and larger spaces pushed the development of the PA system through the 1950s and 60s, channel counts gradually increased and amplifiers got bigger''.
            \item ``Very few venues had in-house PA systems, so bands tended to travel with their own or hired them locally. It was either set up on stage and operated by the band, much like a guitar of bass amp, or placed nearby and operated by a technician''.
            \item Mixing consoles had already been established as the control surface of choice for recording and broadcast applications, but the design took a little while to coalesce into the form that we're familiar with now.
            \item A key milestone was reached in 1958 when EMI's Record Engineering Development Department installed the first dedicated stereo mixing system at Abbey Road Studios in London, the REDD 17.
            \item The legendary console, used by the Beatles and many others, is widely acknowledged to be the first console that established the template that all subsequent consoles would follow, namely a bank of input channels (8) arranged in rows with a fader at the bottom and EQ controls above.
            \item But the consoles employed in recording studios were unwieldy beasts resembling large, thick tables with two solid legs, they were clearly not suited to the demands of live sound.
            \item Going into the 1970s, the design of analog consoles settled into the basic template of a bunch of input channels and a master section arranged on a near horizontal control surface.
            \item Each channel typically featured, from top to bottom: gain, EQ, auxiliary sends, pan pot and a fader which were ordered that way simply because that was the order of the signal flow (with the obvious exception of post fade sends).
            \item This neat design allowed channel strips to be narrow which enabled more channels to be fitted into smaller frame sizes increasing the channel count and creating the familiar sea of knobs.
            \item A key part of the design of the analog console was the humble fader. Fader like controls already existed but much like the ones found on those pioneering REDD consoles they were actually rotary controls arranged side on and fitted with a lever which extended up to the control surface --- this explains that characteristic curved path of movement.
            \item The person who pioneered the use of the linear fader was Atlantic Records engineer and producer Tom Dowd. He observed that when mixing with knobs you could only manipulate two at once (i.e., one per hand) whereas he was a pianist who wanted to manipulate many channels at once, like the keys of a piano, so he found a company that made linear potentiometers, fitted them to the console in Atlantic Studios and the rest is history.
            \item Digital consoles are basically computers which sample and digitally manipulate the audio in real time before converting the outputs back to analog signals for amplification. Designers soon found themselves freed from the constraint of one control per function and the need to conform to the order of the signal flow and developed a variety of new and unique designs.
            \item A key realization was the fact that once the audio is in the digital domain, it's relatively easy to manipulate in various ways so it was completely logical to bring the outboard on board. This not only made everything compact and portable it also made setting up quicker and easier by removing the need for the interconnection of equipment and eliminating the risk of faults arising from incorrect connection or faultyz cables.
            \item Many of those early digital consoles maintained the analog maxim of a fader for every channel, possibly in a bid to not alienate analog users, but once layer switching was embraced, prodigious channel counts quickly became possible on modest sized surfaces. To cap it all off there was the revelation of total recall, a vital feature which we now take so much for granted that it's hard to imagine a time when you couldn't store and then recall your mix in every exact and minute detail --- days, weeks, even years later.
            \item It's easy to look back now and acknowledge the indomitable rise of digital consoles, but at the time launching a new console involved significant research and development as well as a high degree of risk. There was clearly a demand for high-end consoles but no one was confident that there was demand in the small and mid-sized markets, due to the fact that analog consoles were still considered faster and easier to use, particularly in applications where voluntary operators were involved (such as houses of worship).
            \item A team at Yamaha confronted these issues when they sat down to design the M7CL console in 2003 (having already experienced success with the high-end PM1D and PM5D consoles). They wanted to build a console that was “easier than analog” (which became a tag line for the finished product) and came up with the CentraLogic interface, which continues to this day in the CL and QL Series as well as the PM10.
            \item What the design team realized is that one of the great advantages of a digital console is the reduced size and weight, so they gathered the controls into the center of the desk and made them logical to operate (hence the name). Essential to the design was a touch screen that cleverly combined information and operation, something which was not at all common (or indeed cheap) at the time.            
        \end{itemize}
        \newpage{}
    \section{Source 4 \cite{coulesHistoryLiveSound2021}}
    \section{Source 5 \cite{coulesHistoryLiveSound2021a}}
        \bigskip
        \begin{itemize}
        
            \item Charlie Watkins was a British audio engineer who established his company, Watkins Electric Music, in 1949. 
            \item During its early years, Watkins's company was known primarily for guitar amps like their unique, V-shaped Dominator. Watkins is also responsible for the famous Copicat tape delay unit, which is still used today by musicians seeking an authentic, vintage echo effect.
            \item During the 1960s, Watkins began experimenting with building custom sound systems with improved frequency responses. He believed that general-purpose loudspeakers had cones that were too stiff, producing bandwidth too narrow for the subtleties of the human voice. After testing out the Goodmans Axiom 301 speaker, he discovered that its softer cone moved more easily than other speakers, producing a flatter frequency response. He combined the Goodmans Axiom 301 speaker with a custom-built amplifier based on an RCA design to create his first live sound system, which was generally considered to be louder than other systems at the time.
            \item Watkins was able to achieve this level of loudness with a unique power solution. Watkins connected a main amplifier to an array of gain-matched amplifiers, and when he adjusted the gain on the main amplifier, it would increase the level of all the amplifiers with a single knob. Watkins debuted his PA at the 1967 Windows National Jazz \& Blues Festival, and it was so loud, he was arrested for disturbing the peace. Thankfully, the judge threw his case out of court and allowed Watkins to continue operating his PA systems.
            \item Around this same time, American audio engineer Bill Hanley was experimenting with his own custom-built PA systems. Not satisfied with common PA speakers of the time, Hanley designed his own speaker boxes by combining Altec Lansing cinema horns with JBL D130 15-inch drivers. In 1965, Hanley provided the sound system for the inauguration ceremony of President Lyndon B. Johnson. That same year, Hanley's system was used at the famous Newport Folk Festival where Bob Dylan stunned audiences with his first electric performance.
            \item In 1969, Woodstock Festival organizers were struggling to find someone who could provide an adequate sound system for their projected audience of 200,000 people. Bill Hanley jumped at the opportunity to showcase what his custom speaker systems could achieve for large concerts. At Woodstock, Hanley built a massive system using his custom Altec-JBL speakers on two levels of scaffolding and augmented the main system with satellite speakers around the festival grounds. Hanley utilized Macintosh and Crown amplifiers to provide an unprecedented 10,000 watts of power to his speakers. Much has been written about the chaos that ensued at Woodstock, but Hanley was proud to state that the only things that didn't fail during the event were the water supply, the stage security, and the sound system.
            \item At the time, it was common for mixers to be built into the amplifier, but as bigger speaker systems needed more amplifiers to drive them, it was necessary to separate the mixing controls from the amplifiers. In 1974, British manufacturer Soundcraft revolutionized the industry with the Series 1, the first mixing console built into a flight case. The Series 1 was available in 12- and 16-channel versions and helped establish the vertical channel design that became universal among analog mixers.
            \item The Soundcraft Series 1 and its successors made another evolutionary step possible: the front-of-house mixing position. Before that time, it was common for the sound to be mixed from the side of the stage. But now that the mixer was separate from the amplifiers, audio engineers could mix from a position in front of the loudspeakers, allowing them to hear the mix the way the audience experienced it. This enabled engineers to deliver better-sounding mixes and established an industry-wide practice that is still observed today.
            \item 1987 also witnessed the birth of another innovation that would change the live sound industry forever: the digital audio mixer. Yamaha created the DMP7: a recallable mixer that enabled keyboard players to manage their increasingly complex array of keyboards and automatically change settings during shows.
            \item As more manufacturers hopped on the digital trend and sound quality improved, sound engineers began favoring digital mixers over analog consoles because it allowed them to save their configurations and deliver more consistent audio mixes night after night. When factoring that digital mixers can handle significantly more processing, advanced routing, and internal effects engines in a smaller design, the dominance of digital consoles became inevitable.
            \item The final innovation that changed the live sound industry came in the early 1990s. Until this point, PA systems were commonly ground-stacked, point-source systems that produced high sound pressure levels near the front of the stage but lost volume over considerable distances. In 1993, Christian Heil of Heil Sound provided a solution with the unveiling of the V-DOSC, the world's first line array speaker system.
            \item Line array systems work on the principle of closely aligned adjacent speaker drivers, which reinforce each other and push the sound further. The benefits of line array systems include more consistent volume levels over distance, wider horizontal dispersion and less vertical transmission, which results in enhanced frequency response balance and loudness throughout a venue. Today, it's almost impossible to imagine attending a festival or stadium concert without seeing the familiar J-shape of the line array.
        \end{itemize}
    \newpage
    \section{Source 6 \autocite{daleyWindingRoadModern2020}}
        \begin{itemize}
            \item But by the turn of the century, digital live-sound consoles were proliferating, led by examples like Yamaha's PM-1D in 2001, Digico's D5 Live in 2002, and the Midas XL8 Live in March 2006. Digital consoles had crossed the great divide of reassurance for live applications, in the process becoming both more featured and easier to use.
            \item A key turning point in this evolutionary progression was the integration of the touchscreen into the work surface of the desk. ``That's what made it possible to put new features within reach onto the work surface'' says Matt Larson, vice president of professional audio products for Digico. He points out that the shift to digital consoles allowed new operational features to be added to the consoles via software, such as saving a performance's settings in a single file, and those features could be accessed in a “soft” manner, without having to be locked to a particular knob, vastly increasing the flexibility of a console's configuration. 
        \end{itemize}
    \newpage
    \section{source 7 \autocite{fletcherOnceTimeEvolution2014}}
    \section{Source 8 \autocite{gillmoreBriefHistoryTheatre2016}}
        \begin{itemize}
            \item As shows became bigger, other areas of automation and being able to save settings for individual scenes became crucial, as operators found they struggled to adjust settings manually for expanding numbers of performers. Bruce and his colleagues were also under pressure from producers to find a smaller alternative for the large analogue consoles that took up valuable seat space.
            \item Back at Cadac, analogue consoles were still being produced with digital automation. However, there was another company with an eye on the digital console market: Soundtracs, producer of analogue consoles for studio and live sound since the 1980s. In 2002 the decision was made to launch a new brand focussed on digital consoles for live events. The new company, DiGiCo, and a new digital console, the D5 Live, were launched almost immediately.
            \item James Gordon, the founder of DiGiCo, had already spotted the potential need in the theatre sound market and approached Andrew Bruce directly to develop theatre sound-specific software for the new DiGiCo console. Three months and a lot of research and development later, they launched the D5T\@: a digital DiGiCo D5 console with ``T'' (theatre) software, specifically designed to meet the needs of large-scale musical theatre shows. As computer automation control changed the game for theatre sound engineers twenty years earlier, theatre-specific software like T-software took it to a whole new level.
            \item Theatre-sound-specific hardware and software like the DiGiCo consoles with T software were developed to meet these challenges. They allow for controlled and accurate line-by-line mixing, where each mic is only live when lines are sung or spoken, by pulling groups of channels into the central VCA section. Control of the channels is placed right under the operator's hands, saving them from to running up and down a long console riding faders for individual channels
        \end{itemize}
    \newpage
    \section{source 9 \autocite{guitarcenterEvolutionRecordingMixing}}
        \begin{itemize}
            \item \textbf{1930s} Following the revolutionary release of “The Jazz Singer” in 1927—the first feature-length movie to use a synchronized pre-recorded music score and lip-synching, studios around the world began purchasing audio recording equipment for their motion picture productions. Western Electric and RCA made the first consoles and leased them to various studios. With locations in New York and Hollywood, both manufacturers specialized in sound recording for film. The first consoles used optical audio designs, vacuum tubes, and were based on designs found in broadcast, sound reinforcement and telephone equipment.
            \item \textbf(1940s) After the end of World War II in 1945, new technologies from Germany and other parts of Europe began to make their way to the United States. In contrast to the leasing business models of Western Electric and RCA, manufacturers like Ampex, Nagra, Rangertone and Fairchild were willing to sell their consoles directly to studios. With the continued growth of the audio console market, Western Electric and RCA remained two of the most popular suppliers for film sound recording equipment, until technology advanced in the 1970s.
            \item The 1950s brought about some of the most significant changes since the introduction of the first consoles, as magnetic tape became the industry standard for recording audio. In 1958, EMI installed the first dedicated stereo mixing system, the REDD 17, at Abbey Road Studios in London. This was the exact, legendary console used by the Beatles and many more, and is often credited with establishing the concept of the large-format console still used today—with faders at the bottom and EQ controls for each channel above.
            \item That same year, the Universal Audio 610 tube console was also invented, offering a ready-made desk for recording studios. As commercial consoles had yet to become commonplace, purchasing one of the earliest 610 desks often meant buying individual modules and building the console on your own. A few UA 610 consoles were actually built by Bill Putnam Sr.\ (the founder of Universal Audio) himself, and included a custom frame, power supply, metering and basic routing capabilities. Some of the biggest artists of the time purchased custom-built 610 desks to be built by Bill Putnam Sr., including Ray Charles and Frank Sinatra.
            \item \textbf{1960s} As technology progressed in the '60s, recording and mixing consoles began to switch from the tube-based designs of the 1950s to newer solid-state electronics. The older tube consoles offered a distinct tone compared to transistor-based designs. When pushed harder, tube-based audio gear generally produces a more rounded, smoother sound with pleasing harmonic distortion. This is part of the magic that makes tube gear popular so many years later. Transistor-based consoles allowed for smaller footprints that generated less heat and had lower power requirements. By 1964, Rupert Neve released the first-ever commercial solid-state recording console.
            \item With solid-state condenser microphones growing in popularity during this time, console designers began to incorporate microphone powering capability into their mixers, which eliminated the need for outboard power supplies. During this era, consoles with up to 24 channels, like the famed TG12345, also became popular to accommodate 8-track recording devices. In 1967, the first operational amplifiers were used in professional audio gear, notably as summing devices for multichannel consoles.
            \item \textbf{1990s} In the 1990s, the digital audio revolution began to take hold with the introduction of Sound Tools, the precursor to Pro Tools. It quickly became popular in commercial studios around the world.
            \item As DAW (Digital Audio Workstation) software became the center of studio workflows for recording and mixing, DAW control surfaces began to rise in popularity. By 1998, the first Pro Tools controller, DigiDesign ProControl, was introduced. It featured eight motorized faders, a meter bridge and an array of hands-on controls that integrated seamlessly with Pro Tools. It made mixing in-the-box more feasible by bridging the flexibility of DAW software with physical faders, advanced automation and controls for AUX sends, mutes and plug-ins.
        \end{itemize}
    \newpage
    \section{source 10 \autocite{smithWayWeWere2018}}
    \section{source 11 \autocite{smithWayWeWere2018a}}
    \section{source 12 \autocite{sweetwaterLegendaryConsolesTheir2019}}
    \section{source 12 \autocite{HistoryStudioConsole2023}}
        \begin{itemize}
            \item The evolution of the console as we know it began in the mid-to-late 1950s, spurred by the advent of stereo and multitrack recording. After perfecting the art of sound-on-sound recording with single-track tape machines, guitarist and engineer Les Paul (AKA the “Wizard of Waukesha”) synchronized eight tape decks to create the first true multitrack recording system for his home studio in Waukesha, Wisconsin. “The Octopus,” as he called it, provided more flexibility and control over the recording process than ever before, but it also presented the need for an equally flexible mixing and monitoring solution. Enter the console.
            \item 1956
            \begin{itemize}
                \item \textbf{Les Paul commissions a custom eight-channel console} Shortly after creating The Octopus, Paul engaged Rein Narma Audio Engineering to build a custom console that could provide individual control and mixing capabilities for all eight of The Octopus' channels. While an impressive feat of engineering, the completely custom-made console only served Paul's specific needs. The concept of the studio console was burgeoning, but it needed some improvements in order to gain widespread use.
            \end{itemize}
            \item 1957
            \begin{itemize}
                \item \textbf{EMI designs REDD.17 console for Abbey Road Studios}Across the pond, EMI's Record Engineering Development Department (REDD) was simultaneously working on its own console for in-house use at the famous Abbey Road Studios. Building on the early REDD.1 system (which doesn't quite qualify as a console), the REDD.17 allowed recording engineers to balance and pan eight microphone inputs while recording directly to a two-track stereo tape machine. The faders were cleverly arranged in two banks of four, allowing one operator to control the levels of all eight inputs simultaneously.
            \end{itemize}
        \end{itemize}
    \newpage
    \section{source 13 \autocite{TheatricalAddressLoudspeakers}}
        \begin{itemize}
            \item ``When we started in 1973, there were no loudspeakers, there were no mixers, there were no radio mics, there were no miniature mics that were made specifically for theatre use'' recalls Autograph Sound Recording's Andrew Bruce.
            \item With Autograph working on theatre sound in the West End and Abe Jacob, The Godfather of Theatre Sound, on the case on Broadway, the pressure on audio equipment manufacturers to recognise the unique requirements of the theatre was mounting at the tail end of the 1970s. Slowly but surely, theatre made its way into their R\x   &D departments and the solutions that emerged earned acceptance from sound designers.
            \item While Meyer Sound's UPA was not specifically a theatre design, it provided many of the features needed for musical theatre production. At the top of the list were sound quality, sound output and size. These needed to be excellent, high and small respectively. At this time, there were essentially three broad categories of speaker: single-box, stacked separates and columns (line arrays).
            \item ``The column was used in theatres until the mid-1980s for both straight plays and musicals in only two incarnations --- the small and large Bozak Column. Both were vertical arrays of mid-frequency aluminium cone drivers and dome tweeters, with twice as many tweeters as mid-frequency drivers. The very narrow, vertical high frequency coverage allowed the speakers to be positioned low on the stage to “skim” the audience in both the orchestra and balcony seats with very even coverage. When used correctly, the results were excellent, limited only by available amplifiers and power handling capabilities. You might call these the parents or even the grandparents of modern line array or curvilinear array systems.''
            \item ``Abe [Jacob]'s goal was to find a product that was small enough to be invisible and had a transparent audio quality'', says Bob McCarthy, Director of System Optimization, who was involved with Meyer Sound from the early stages of the company. ``Abe was drawn to the audio signature of the UM-1, which had superb intelligibility for speech but extremely tight coverage because it was intended for musicians onstage. Abe wanted John Meyer to build a flyable version with a wider pattern. This led to the development of the UPA, a staple that remains one of the best-selling products from Meyer Sound, with more than 30,000 shipped''.
            \item ``The UM was Meyer Sound's first product and was considered revolutionary, mainly because it was designed from the ground up for consistency. Dedicated electronics, crossovers, and presets delivered very consistent performance from box to box, and from show to show. This was very different from the common approach at the time. In the late-1970s, when you set up a system for a show, you could pretty much expect some parts of it to fail in some way''.
            \item ``The UPA loudspeaker was an ideal theatre speaker'', Andrew Bruce offers. ``John and Helen Meyer knew that the market was there for developing theatre speakers --- small and powerful because theatres have small proscenium arches where traditional loudspeakers just don't work''.
            \item Bruce regards only two manufacturers to have been taking theatre sound seriously at this time, Meyer Sound with loudspeakers and Cadac with mixing consoles. ``We had to learn how to persuade manufacturers to see things our way'', he says.
            \item Among those who were persuaded, the big three manufacturers for single-source speakers were to be Meyer Sound, Eastern Acoustic Works (EAW) and Apogee Sound through the 1980s. Each offering a range of speakers delivering higher power and using lighter amplifiers, these were the backbone of theatre audio until mid-size line arrays appeared around 2000.
            \item Of these, the L-Acoustics V-Dosc system was used for the Broadway production of Mamma Mia! in 2001 and for the first year of its US tour in 2002. The advent of mid-size and smaller line array systems led to them becoming the primary choice, especially for musicals. The big three manufacturers for column or line array speakers remain Meyer Sound (M'elodie and Mina), d\&b audiotechnik (Q, T, V series) and L-Acoustics (dV-Dosc and Kara). In particular, the dV-Dosc speaker that was introduced in 1999 is still in use today.
        \end{itemize}
    \nocite{*}
\newpage
\printbibliography[]


\end{document}