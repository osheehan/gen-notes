\documentclass[12pt]{article}
\usepackage[letterpaper, portrait, margin=1in]{geometry}
\usepackage[hidelinks]{hyperref}
\usepackage{setspace}
\usepackage{fancyhdr}
\usepackage{graphicx}
\usepackage[T1]{fontenc}
\usepackage{mathptmx}
\pagestyle{fancy}
 

\fancyhf{} % sets both header and footer to nothing}
\renewcommand{\headrulewidth}{0pt}
\pagenumbering{arabic}
\rhead{Sheehan \thepage} 

\def\class{Production Resource Management}
\def\prof{Shaye Sahoo \& David Holcomb}
\def\due{01/23/25}

\fancypagestyle{1stPage}{
    \fancyhf{}
    \renewcommand{\headrulewidth}{0pt} % removes horizontal header line
    \lhead{Owen M. Sheehan \\ \prof{} \\ \class{} \\ \due{}}
    \rhead{ Sheehan 1}
}

\thispagestyle{1stPage}
\begin{document}
\begin{doublespace}
\vspace*{20pt}
\begin{center} \textbf{Pre-Production Analysis} \end{center}


\subsection*{Overall Production}
    \par The production in question for this analysis is Dominique Morriseau's \textit{Detroit '67}, which was written in 2013, and is set in 1967 in Detroit, during the 1967 Detroit Riot.
    \par Regarding the production as a whole the proposed budget is is \$25,832, with there being very little special considerations. The show has 5 characters, with each character having about 2 costumes. There is also only one full set that is used through 11 scenes over 2 acts.
    \par For me, this play is similar to both \textit{Put Your House in Order} and \textit{Lonely Planet} in the sense that it has a pretty small cast that is reacting to events that happen offstage. It is also similar to \textit{John Proctor is the Villain} in that the music and sound design throughout the show is a driving force in the show.
    \par Now we're going to go through each department one-by-one.

\subsection*{Scenic}
    \par First, we're going to look into scenic, as it will take up the largest portion of the budget, with about \$12,000. 
    \par The Script calls for a set that is made to look like the unfinised basement of a two-story home in Detroit. The biggest obstacle will probably be the balcony and stairs that lead into the balcony.
    \par as with most productions Scenery will probably have to work extensively with props and somewhat lighting. This set calls for extensive set dressing which will be covered later in the props section.
    \par In my mind, the most likely way to achieve what the script calls for set-wise, is a box set, especially because the script specifically calls out details for the walls (which will be covered in the paints section), and the fact that there is windows for sunlight to come through.
    \par Also of note is that the set is static throughout, there is not set changes, however, the set dressing does slightly change throughout, however that is more of props problem than scenic's
\subsection*{Paints}
    \par The next department is paints, which will probably deal with their usual fare, i.e.\ wall and floor traeatments, as well as specific wall detailing.
    \par In specific, if the floor gets treatment, it will most likely be detailed to look like a concrete floor, and the walls will most likely be made to look like cinder block walls, with the specific markings called out in the script which are as follows: drawing of a star, black fist, and a ``lumpy'' portait of a girl (young michelle).
    \par The only specific considerations for paints is they will most likely have to coordinate wit the props department for a few props, most importantly the Velvet painting that Lank and Sly bring back to the house.
    \par The Paints budget estimation is \$3,000.
\subsection*{Props}
    \par There are a ton of props in the show, mainly because the script is written in a mostly realistic way causing there to be a ton of things that characters interact with. 
    \par There are a few props that will require coordination with other departments, mostly sound, due to the record player, 8-track player, and the speakers/music system, especially if it is decided that these need to be operational. Paints may be involeved in the velvet painting. Depending on if the christmas lights or neon light sign needs to be functional, lighting may ned to be invloved.
    \par The following is a list of large props: Cabinets and board making up a bar, freezer, washer/dryer, sink, clothes line, couch, recliner, pipe poles (floor to ceiling), crates covered in cloth, stool, and a fuse box.
    \par The following is a list of small props: record player and records, 8-track player and tapes, christmas lights, boxes, velvet painting, variou party favors, speakers/music system, clothes (laundry), frozen peas, towel, ointment, bandages, pillows, blanket, money, extension cord, flashlight, batteries, cloth, broom and duster, newspaper, and drinking glasses.
    \par There is also a few props that are edible which are as follows: toast with jam, nuts, [fake] liquor, a bowl of punch, and a bag of chips.
    \par The proposed budget for Props is \$3,355
\subsection*{Costumes}
    \par Costumes for this show can go, in my mind, two different directions when deciding how many costumes w, either imply costume changes, or only do costume changes when it is specifically called out (I guess this is true of most shows, but still).
    \par When it comes to perscribed constume changes, there is caroline changing clothes after Chel gives her new clothes to change into, then Sly coming in a new suit, as well as both Sly and Lank coming in disheveled and bloodied at different points.
    \par The proposed budget is \$5,477 with each costume being allocated \$547, assuming that each character has 2 costumes.
\subsection*{Lighting}
    \par Next up is lighting which has a budget of \$1,500, which is in line with other lighting budgets for CMU\@.
    \par Important considerations for lighting is that this show has a day/night cycle, with about 3/5\textsuperscript{ths} of the show happening during the night. The script also calls out that there is light shining through windows during one of the scenes that takes place during the morning.
    \par Another consideration is dependent on how scenic contructs the set, if they do something like what they did in \textit{Proctor}, it could be quite difficult to get the kind of angles they need, especically if they opt for a proscenium to make a box set sans ceiling as that probably wouldn't work very well in the Rauh.
\subsection*{Sound}
    \par The last department that I'm going to talk about it Sound, like every other show that isn't the musical, the budget for this show will be \$500.
    \par The most important aspect regarding sound is that the show calls out a tone of music that should be used in the course of the show, and these specific tracks may need to be processed to sound like they are playing on vinyl/8-track, as well as addind the skipping for the tracks on vinyl.
    \par Other considerations for sound are that there shouldn't be a need for close micing actors or for live musicians. However, they may be a need for composition for transitions, as well as sfx sourcing for sounds such as the tanks rolling above the basement.


\end{doublespace}
\end{document}