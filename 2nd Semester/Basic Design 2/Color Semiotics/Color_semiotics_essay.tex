\documentclass[12pt]{article}
\usepackage[letterpaper, portrait, margin=1in]{geometry}
\usepackage[hidelinks]{hyperref}
\usepackage{setspace}
\usepackage{fancyhdr}
\usepackage{graphicx}
\usepackage[T1]{fontenc}
\usepackage{mathptmx}
\usepackage[backend=biber, style=mla]{biblatex}
\pagestyle{fancy}

\addbibresource{colorSemiotics.bib}    

\fancyhf{} % sets both header and footer to nothing}
\renewcommand{\headrulewidth}{0pt}
\pagenumbering{arabic}
\rhead{Sheehan \thepage} 

\def\class{Basic Design 2}
\def\prof{Olivera Gajic}
\def\due{1/23/24}

\fancypagestyle{1stPage}{
    \fancyhf{}
    \renewcommand{\headrulewidth}{0pt} % removes horizontal header line
    \lhead{\href{mailto:osheehan@andrew.cmu.edu}{Owen M. Sheehan} \\ \href{mailto:ogajic@andrew.cmu.edu}{\prof{}} \\ \class{} \\ \due{}}
    \rhead{ Sheehan 1}
}

\thispagestyle{1stPage}
\begin{document}
\begin{doublespace}
    \vspace*{20pt}
        \begin{center}
            \textbf{Color Semiotics: Green}
        \end{center}

        \par In this short essay I will discuss my findings with regard to the semiotics of the color green.
        The best way to structure these findings is to go over each source one-by-one.
        \par 
            The first source I came upon was an article written by \citeauthor{ModernMet} for \href{www.mymodernmet.com}{\textit{My Modern Met}} called \citetitle{ModernMet}. In said article \citeauthor*{ModernMet} states that green is most commonly associated with nature and sustainability and eco-friendliness today. 
            She then goes on to list the uses of green in history saying that in the Middle Ages, clothing was used to indicate a person's social rank and profession with green signifying upper class professions such as merchants, bankers, and the gentry.
            She also explains how despite the fact green pigments are usually toxic, green is still associated with feelings of vitality, freshness, calmness, and revival.
        \par
            The next article I stumbled upon, to me, was the most interesting of the bunch. This article is called \citetitle{Islam} and is written by \citeauthor{Islam}.
            To me, it is quite interesting that green symbolizes so many things in Islamic/Iranian culture that are mostly different compared to where I am from, which is mostly protestant Christians. The authors claim that ``green is the symbol of [the] sky, farmer[s] and agriculture, justice, angel[s], heaven and heavenly beings, [the] Prophet and imams, [the] moon and water, wisdom and knowledge, faith and firm belief, vitality and youth and \dots mysticism and Sufism'' \autocite[26]{Islam}.
        \par 
            The last article I really looked at was \citetitle{colorSymbolism} written in 1919 by Dr.\ \citeauthor{colorSymbolism}. 
            This article is also quite interesting as it tries to list different characteristics of most colors, along with historical and cultural meanings. If I'm being honest a majority of my notes were based off of this article. 
            According to \citeauthor*{colorSymbolism}, ``green c[a]me to be the emblem of freshness, of youth, of growth, of regeneration, [and] of activity. It is the visible manifestation of the productive union of earth and water.'' \autocite[139-140]{colorSymbolism}.
             He goes on to say that the Greeks used green to represent love and the ocean, due to green's association with Aphrodite and the sea foam that she spawned from \autocite[140]{colorSymbolism}.
             When it comes to how the Chinese uses green, \Citeauthor*{colorSymbolism} states that ``In China we find that green was symbolic of the east, of a tree, of the spring, \dots of charity, and of regeneration'' \autocite[140]{colorSymbolism}. 
             When it comes to the early Christian Church, green represented ``charity, regeneration, and hope'' \autocite[140]{colorSymbolism}. 
        \par While green usually has a positive meaning this isn't always true. Apparently, to the Greeks, green symbolized both victory and defeat and flight \autocite[Evarts 141]{colorSymbolism}. While green is regularly used to represent both luck and the Irish, with \citeauthor*{colorSymbolism} claiming that ``green is loved by the Irish in whatever corner of the world they may be, because it is emblematic of their own Emerald Isle'' \autocite[141]{colorSymbolism} Which to me reads as very early 20\textsuperscript{th} century. Despite this, apparently to ``some clans of Scotland[,] green is considered unlucky'' \autocite[141]{colorSymbolism}

\newpage \printbibliography
\end{doublespace}
\end{document}