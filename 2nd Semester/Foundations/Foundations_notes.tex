\documentclass[12pt]{article}
\usepackage[hidelinks]{hyperref}
\usepackage{fancyhdr}
\pagestyle{fancy}
\fancyfoot{}
\rhead{\thepage}
\renewcommand{\headrulewidth}{0pt}
\pagenumbering{arabic}

\def\class{Foundations of Drama 1}
\def\prof{Ryan Prendergast}

\title{\class{} Notes}
\author{Owen M. Sheehan\\Professor \prof{}}
\date{Spring 2024}

\begin{document}
\maketitle
\tableofcontents
\newpage

    \section{Week 1}
        \subsection{Thursday 1/18/24}
        \bigskip
            \begin{itemize}
                \item Dramaturgy
                \begin{itemize}
                    \item How dramatic texts are constructed and function
                    \item Underlying principles
                    \item Form and action
                \end{itemize}
            \end{itemize}
    
    \section{Week 2}
        \subsection{What is research}
            \begin{itemize}
                \item Research is thought, conversation, and community
                \item ethical practices in research are central to theatrical practice
                \item we are here because of research
                \item ``We do reseatch whenever we gather information to answer a question that solves a problem'' (Craft, 10)
                \item broad to specific
                \item Due diligence
                \item Always ask questions, ``who cares?''
                \item challenges to research
                \begin{itemize}
                    \item Manageable scope
                    \item Cherry-picked evidence
                    \item starting with an answer and not question
                    \item ``sunken cost fallacy''
                    \item when to end
                    \item scope creep
                \end{itemize}
                \item differentiate between
                \begin{itemize}
                    \item report
                    \begin{itemize}
                        \item informs the reader
                        \item factually answers questions
                        \item summarizes existing research
                    \end{itemize}
                    \item Position/opinion paper
                    \begin{itemize}
                        \item persuades the reader
                        \item uses rhetoric to convince
                    \end{itemize}
                    \item Research paper
                    \begin{itemize}
                        \item contributes new knowledge and/or perspectives
                        \item adds to the conversation of research
                        \item considers multiple perspectives
                        \item supports with detailed evidence
                    \end{itemize}
                \end{itemize}
                \item what makes arts \& humanities reseatch unique 
                \begin{itemize}
                    \item ephemeral nature of the object(s)
                    \item the reliance on interpretation
                    \item the creative process follows unorthodox logic(s)
                    \item the quality of materials
                \end{itemize}
                \item Research materials --- Primary, secondary, and tertiary sources
                \begin{description}
                    \item[Primary Sources] immediate connection to a movement, event, or period concerned
                    \item[Secondary Sources] Analyze and interpret primary sources at a level removed from them
                    \item[Tertiary Sources] Compile data on topics  
                \end{description}
                \item Who has primary materials?
                \begin{itemize}
                    \item Libraries
                    \item Archives
                    \item Museums
                    \item Image Services
                    \item Private collections
                \end{itemize}
                \item Acceptable sources
                \begin{itemize}
                    \item Sourced from a library, scholarly hub, or database
                    \item produced and/or endorsed by a recognized and/or authoritative source
                    \item peer reviewed
                    \item shows a clear provenance
                \end{itemize}
                \item Unacceptable sources
                \begin{itemize}
                    \item anything without a clear attribution of author or publisher (no random sketchy PDFs)
                    \item unvetted \& .com search engine hits
                    \item wikipedia (go to sources tab)
                    \item if unsure, ask
                \end{itemize}
                \item Citing Sources
                \begin{itemize}
                    \item central to ethical research
                    \item Varies by discipline and publication --- MLA for theatrical research
                    \item You owe it to the sources who came before you to cite them, and it shows your contribution to the ongoing ``conversation''
                    \item Non-citation can lead to claims of plagiarism (don't do it)
                \end{itemize}
            \end{itemize}
            \newpage
    \section{Week 3}
        \subsection{Reading}
            \subsubsection{Break out groups}
                \begin{enumerate}
                    \item What constitutes/makes/qualifies a text as a ``dramatic texts''
                    \begin{itemize}
                        \item Something that is intended to be performed
                        \item works can be a dramatic text loosely
                    \end{itemize}
                    \item What gives any dramatic text ``authority''? Who controls a dramatic text?
                    \begin{itemize}
                        \item The act of it being performed -iffy
                        \item whoever is performing it
                        \item but the author can have notes on how its supposed to be performed
                    \end{itemize}
                    \item What do we value and devalue in dramatic texts over time?
                    \begin{itemize}
                        \item We value emotional relevancy
                        \item we value social relevance over historical accuracy
                        \item we devalue specific details
                    \end{itemize}
                    \item How do we read in our field
                    \begin{itemize}
                        \item If reading for enjoyement, going to focus on dialouge
                        \item if designing, stage directions are most important
                        \item Evie analyzes everything
                    \end{itemize}
                \end{enumerate}

\end{document}