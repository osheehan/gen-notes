\documentclass[12pt]{article}
\usepackage[letterpaper, portrait, margin=1in]{geometry}
\usepackage[hidelinks]{hyperref}
\usepackage{setspace}
\usepackage{fancyhdr}
\usepackage{graphicx}
\usepackage{setspace}
\usepackage[T1]{fontenc}
\usepackage{mathptmx}
\usepackage[backend=biber, style=mla]{biblatex}
\pagestyle{fancy}

\addbibresource{reading_review.bib}    

\fancyhf{} % sets both header and footer to nothing}
\renewcommand{\headrulewidth}{0pt}
\pagenumbering{arabic}
\rhead{Sheehan \thepage} 

\def\class{Modern Ireland: Politics and Culture from the Famine (1847) to Today}
\def\prof{Dr.\ Aiden Beatty}
\def\due{2/16/2024}

\fancypagestyle{1stPage}{
    \fancyhf{}
    \renewcommand{\headrulewidth}{0pt} % removes horizontal header line
    \lhead{Owen M. Sheehan \\ \prof{} \\ \class{} \\ \due{}}
    \rhead{ Sheehan 1}
}

\setstretch{2.10}
\thispagestyle{1stPage}
\begin{document}

\vspace*{20pt}
    \begin{center}
    \large{Reading Review}
    \end{center}
    \smallskip
% 5) What is Irish Nationalism? Is it simply an ideology of national liberation or is it something else?
% 
% Choose any one reading relevant to your paper from the class readings and write a 3-page review. Why
% did you choose this reading? How does it help you to answer your chosen question? What are the central
% arguments of the reading? Do you agree with them? Deadline – Friday of Week 5 (16 February) – 20%
    \par For this reading review, and the independent paper as a whole, I have decided on the question of  What is ``Irish Nationalism? Is it simply an ideology of national liberation or is it something else?''
    To help get more info regarding this question I read \textit{The necessity for de-Anglicising Ireland}.
    \par The article has a quite interesting premise, on a high level it talks about how the Irish people were essentially losing their identity due to the Anglicization of Ireland.
    The reason I chose this article I think is quite clear. It's hard to talk about nationalism, when a national identity doesn't exist. To have ``pride in one's nation'', that nation has to have some unique identity.
    When looking at Ireland in the 1800s, I think it's quite clear that England did as England does, and just like many other lands that England ``Colonized'', the local populace was forced to assimilate into English culture.
    \par That being said, I'd like to cover the specific points made in the article. The first and main point being the loss of the ubiquity of the Irish language.
    This point is heavily belabored and makes up a majority of the article, and it makes sense why. A major part of any culture, is the shared language, we rely on language so much that I couldn't imagine what it would be like to grow up seeing a foreign country essentially eliminating the language you speak.
    \par I do want to take a minute to talk about a part of the article that I found quite weird. In the part where Hyde compares Irish and English names, he states that Irish names are beautiful and that English names were, in a word, ``dirty''.
    I think this has a dangerous side effect, that may have been intentional on Hyde's part. It demonizes the English language to such an extreme extent, thus alienating the Irish from the English. I think this can lead to a strong sense of unneeded remorse in the people who had already changed their name.
    I think this can have a negative impact on the population that you're trying to convince to de-Anglicanize, especially when Hyde himself pointed out that a large part of the population had taken English names.
    \par This criticism is purely hypothetical though, since if you were pushing for a free state, you wouldn't necessarily need the approval of the people, you would just need to convince the British Government through either revolution or policy.
    \par Another point that Hyde brings up is the formalization, and strict implementation of the teaching of the Irish language as opposed to English. I don't think that that would have really gone over that well. Once a majority of a nation speaks one language, it is incredibly hard to switch that national language. Take for example, the amount of people who can speak Irish in the modern day is shockingly low.
    \par I think for the most part I do agree with Hyde that a people's identity stems from a shared experience, and that having a shared language is a large part of this. However, some of the ways that Hyde goes on to ``prove'' his points slightly irks me.
    Now, some of these issues stems from the changes in how we, as a society, speak in general as opposed to the late 1800s. So, for that, I can leave some leeway.
    \par I think this article also shows the beginnings of systematized nationalism in Ireland (however, I could be very wrong in my mental timeline, so take this with a grain of salt), the argument that the Britainization of Ireland diluted the Irish identity can, in my opinion, not really be disputed.
    \par All in all, this article brings up many good points, even if in slightly weird ways. It is an important step in understanding nationalism in Ireland, especially when it comes to home rule.





\end{document}