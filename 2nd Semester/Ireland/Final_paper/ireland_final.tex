\documentclass[12pt]{article}
\usepackage[letterpaper, portrait, margin=1in]{geometry}
\usepackage[hidelinks]{hyperref}
\usepackage{setspace}

\usepackage{fancyhdr}
\usepackage{graphicx}
\usepackage[T1]{fontenc}
\usepackage{mathptmx}
\usepackage[backend=biber, style=mla]{biblatex}
\pagestyle{fancy}


\addbibresource{ireland_final.bib}    

\fancyhf{} % sets both header and footer to nothing}
\renewcommand{\headrulewidth}{0pt}
\pagenumbering{arabic}
\rhead{Sheehan \thepage} 


\def\class{Modern Ireland: Politics and Culture from the Famine (1847) to Today}
\def\prof{Dr.\ Aidan Beatty}
\def\due{05/02/24}

\fancypagestyle{1stPage}{
    \fancyhf{}
    \renewcommand{\headrulewidth}{0pt} % removes horizontal header line
    \lhead{Owen M. Sheehan \\ \prof{} \\ \class{} \\ \due{}}
    \rhead{ Sheehan 1}
}

\thispagestyle{1stPage}
\begin{document}
\setstretch{2.5}
\vspace*{20pt}
\begin{center}
    \textbf{Final Paper: ``What is Irish Nationalism? Is it simply an ideology of national liberation or is it something else?''}
\end{center}
\bigskip

% #region intro
    \par My research paper is focused on the question of what is Irish nationalism. I personally do not have a concrete answer to this question as that would require much more research than what is needed for this paper.
    I will, however, pick a concrete position to this question for the sake of argument. 
% #region Par 1
    \par Nationalism can be divided into multiple different categories. The main divisor that comes to mind is that if one referring to Nationalism, do they mean it in a democratic sense or in a totalitarian sense. 
    This sentiment can be easily seen, take for example, Germany. At different points in Germany's history, multiple different versions of nationality can be seen.
    At the beginning of the formation of the German States, nationalism in the sense of the forming of a nation can be seen. However, during the world wars a different form of Nationalism takes place. That being, the sense that one's nation is the best nation. This can easily be the most dangerous form of nationalism as it can swiftly lead to totalitarian ideals.
% #region Par 2
    \par I don't think Irish Nationalism easily falls into these two categories, as nationalism can be seen in yet another light, that being the argument between those that wanted to be part of the United Kingdom (Loyalists) and those who wanted to become an independent country (Nationalists).
% #region Par 3
    \par I believe the tensions really started in January 1801, when ``[Ireland] \dots entered the United Kingdom with most of its people as second-class citizens'' \autocite[4]{Components}.
% #region Par 4
\par There were many different ways that people dealt with this tension, some like the Ulster United Irishmen responded with violence. But others like Daniel O'Connell believed that violence could ``release the passions of ignorant peasants made desperate by Tyranny and injustice. They would fight and lose after a bloody slaughter. Their enemies, at they did with the Act of the Union, would use failed'' \autocite[5]{Components}.
% #region Par 5
    \par In 1870, the Home Rule movement was started by Isaac Butt. This movement fought for the establishment of a self-ruling Irish nation under the United Kingdom, with Butt believing that ``a self-governing Ireland would enthusiastically share the burdens as well as the glories of the British Empire'' \autocite[8]{Components}.
    \par Butt's movement excluded agrarian and Catholic issues and hence did not gain widespread appeal, whereas Charles Stewart Parnell made Home Rule popular by ``combining demand for an Irish Parliament with a war on landlordism. Still, he did not abandon the inclusive ideology of Irish nationalism. He said that Ireland needed the talents of all classes and creeds'' \autocite[8]{Components}
% #region Par 6
    \par One of the biggest points in all of Irish politics, was that of land ownership. This also applies to nationalism, as one of the problems Nationalists were having was that a lot of land in Ireland didn't belong to Irish people, but rather to British landlords.
    This had many negative effects, with one of them being the fact that since it was only landowners that were involved in politics, there was essentially no representation of the common man in the Irish Parliament.
    The existence of this problem would be detrimental to any nationalist movement, as they almost always rely on the wants of the common man to unite the nation into one state. This problem was tackled in many different proposed Home Rule bills.
% #region Par 7
    \par From everything that I've read so far, it seems that Nationalism in Ireland mainly refers to the movement of creating an independent Irish nation.
    A lot of the fervor for Home Rule was also helped by Irish-Americans who had a strong national identity once they immigrated, which contributed to nationalistic ideals in the Homeland. 
    \par While Ireland today has quite a strong national identity, this is only spawned by Irish abroad in around the 1920s, before this, Ireland was mostly written off by the rest of the world as a country full of farmers with no culture.
    This spawning of Nationalism from the diaspora led to an influx of money going to nationalistic causes that helped to further the cause.
% #region End
    \par I think I can confidently believe that Irish nationalism is simply an ideology of National Liberation.
    Based on what I've read, while there may be fringe groups that aren't just concerned with national liberation, the vast majority, however, do focus on Liberation and Home Rule. 
    \par I do want to make a note that I most definitely don't have all the data to back up this claim. When it comes to a short research paper, there just isn't enough time or space to cover every outlier. I'm sure there are papers and groups that I just missed, or are buried in books that I didn't go through.
    \newpage
\nocite{*}
\printbibliography
\end{document}