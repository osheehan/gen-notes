\documentclass[12pt]{article}
\usepackage[hidelinks]{hyperref}
\usepackage{fancyhdr}
\pagestyle{fancy}
\fancyfoot{}
\rhead{\thepage}
\renewcommand{\headrulewidth}{0pt}
\pagenumbering{arabic}

\begin{document}
\tableofcontents
\newpage
    \section{Week 1}
        \subsection{Wednesday 1/17/24}
            \bigskip
            \begin{itemize}
                \item We study history through geography which can be messy due to \\changing borders
                \item Ireland is a simple, single geographical area
                \item The island is split between Northern Ireland (UK) and the Republic of Ireland (EU)
                \item Northern Ireland
                \begin{itemize}
                    \item The North of Ireland
                    \item The North
                    \item Ulster
                    \item The Six Counties
                \end{itemize}
                \item Ireland
                \begin{itemize}
                    \item The Irish Free State/Saorstát Éireann
                    \item Éire
                    \item The Republic of Ireland
                    \item The Irish Republic
                    \item The South
                    \item The 26 Counties
                \end{itemize}
                \item The further west you go, the less dense the population becomes
                \item Ireland is officially bilingual
                \item Who and Where and the ``Irish Nation''
                \begin{itemize}
                    \item The Irish nation is a quasi-geographical term, but it resists any simple geographical definition
                    \item And we are not studying ``Ireland'', the geographical space, we are studying ``Ireland'', the demographic, social, economic, political, and cultural phenomenon
                    \item ``Irish'' history is inseparable from ``British'' history
                    \item By 1860, the most Irish city in the world, by population, was New York
                    \item today $\frac{1}{5}$ of the population of ``Ireland'' were not born on the island of Ireland
                    \item Ireland, for the entirety of our period, was a part of global \\capitalism, part of broader trends in European and global politics and culture
                    \item ``Ireland'' is fluid and messy
                \end{itemize}
            \end{itemize}




\end{document}